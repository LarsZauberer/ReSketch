\chapter{Theoretische Grundlagen}

\section{Machine Learning}
\label{chap:t_ml}
Teilbereich künstliche Intelligenz (Künstliche Intelligenz ist Maschine, die menschliche Tätigkeiten nachmacht) Maschine = Computer -> Computerprogramm
Machine Learning ist Überbegriff -> Computerprogramme, die eine Mustererkennung entwickeln. Durch Analysieren von Daten. KI lernt Vorhersagen

Beispiel Zahlenerkennung:
Beispiel für Künstliche Intelligenz
Bild einer handschriftlichen Zahl wird Programm übergeben (Input)
Das Programm bestimmt, um welche Zahl es sich handelt.
Programm gibt Zahl aus

Künstliche Intelligenz weil das Erkennen von Zahlen menschlich ist.

Unterschied machine Learning: KI lernt von Grund auf, Zahlen zu Erkennen
Anfang wird allermeistens Falsch sein

KI lernt durch die Analyse von Daten. -> Trainingsdaten. Dabei ist bekannt, was die richtige Lösung zu den gegebenen Daten ist.
So erfährt die KI, wenn sie falsch liegt. 
Programm kann sich bezogen auf Erfahrung selbst anpassen. Anpassungen möglichst so, dass Vorhersagen in Zukunft besser sind.

Nachdem KI mit Trainingsdaten verbessert wurde, sollte diese auch für neue Daten möglichst genaue Vorhersagen machen.

Nächste Frage: Wie trifft KI entscheidungen, und wie kann sie sich selbst Anpassen? -> Antwort Neuronale Netze


\subsection*{Neuronale Netze}




\section{Reinforcement Learning}
\label{chap:t_rl}
Verschiedene Arten


\section{Verwandte Arbeiten und Themen}
\label{chap:t_verwandt}

Ein verwandtes Thema dieser Arbeit ist die Robotik. Ein Roboter ist laut der
Definition eine `Apparatur, die bestimmte Funktionen eines Menschen ausführen
kann' (Duden). Das Nachzeichnen von Strichbildern ist ebenfalls eine menschliche
Tätigkeit. Diese Arbeit untersucht allerdings nur das Computerprogramm, welches
diese Tätigkeit verrichten kann. Die Apparatur, die von diesem Programm
gesteuert werden könnte, wird also nicht erbaut.

Es gibt verschiedene Ansätze, um ein Computerprogramm die menschliche Tätigkeit
des Nachzeichnens verrichten zu lassen. Ein häufiger Ansatz ist "Stroke-Based
Rendering", wobei Bilder durch das Platzieren von Elementen wie Strichen
gezeichnet werden. Beispiele für Arbeiten, die diesen Ansatz verwenden sind\dots
Stroke-Based Rendering unterscheidet sich von menschlichem Zeichnen dadurch,
dass kein Stift geführt wird. Stattdessen können die Elemente zu jedem Zeitpunkt
an einer willkürlichen Position auf der Zeichenfläche platziert werden.

Andere Ansätze simulieren die Führung eines Stiftes. Damit ist gemeint, dass das
Computerprogramm nicht zu jedem Zeitpunkt an jedem Ort Zeichnen kann.
Stattdessen muss das Programm einen virtuellen Stift bewegen und kann nur gerade
dort zeichnen, wo sich der Stift befindet. Das ist eine Einschränkung, die auch
auf menschliches Zeichnen zutrifft. Ein Beipiel für diese Art des Zeichnens ist
Doodle-SDQ. Das Computerprogramm dieser Arbeit basiert auf dem Programm von
Doodle-SDQ 

\subsection*{Doodle-SDQ}
Doodle-SDQ ist ein Programm, das durch Deep Q Learning erlernt hat, Strichbilder aus dem Google Quick-Draw Datenset nachzuzeichnen. 
Ist in der Freiheit des Zeichnens eingeschränk\indent 

Architektur: Agent = Stift. Umgebung = Zeichenfläche. Speichert Position von
Agent, das zu Nachzeichnende Bild, das was bis jezt gezeichnet wurde und ob
Stift gerade vom Blatt gehoben ist. Die Belohnung/Bestrafung ist der Grad der
Ähnlichkeit zwischen Bildern. Die Unmittelbare Umgebung des Agenten, in dem er
sich in einem Schritt bewegen kann, wird in das neuronale Netz noch ein weiteres
Mal eingegeben, wodurch darauf ein Fokus gelegt wird. Ausgabe des neuronalen
Netz ist eine der Aktionen die der Agent ausführen kann




\section{Git und GitHub}
\label{chap:t_git}