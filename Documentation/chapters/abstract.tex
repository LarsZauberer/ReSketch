\begin{abstract}
    Diese Arbeit beschäftigt sich mit dem Nachahmen des menschlichen
    Nachzeichnens von Strichbildern durch ein Computerprogramm. So wird eine
    künstliche Intelligenz verwendet, um diese Tätigkeit zu erlernen.

    Da es keinen wirklich vorgegeben Weg gibt, wie man ein Strichbild zeichnet
    verwendet die KI Reinforcement Learning um selber einen Weg zu erlernen. Die
    Architektur des Neuronalennetzes basiert grössten Teils auf der Arbeit
    `Learning to Doodle with Deep Q-Networks and Demonstrated Strokes'
    \cite{zhou_learning_2018} Die Arbeit untersucht weiter die Messbarkeit, wie
    gut gezeichnet wurde, die Auswirkung einer physik Umgebung und wie gut die
    Zeichnungen erkannt werden können.

    % TODO: Results und etwas diskussion

\end{abstract}