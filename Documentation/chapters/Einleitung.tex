
\chapter{Einleitung}\label{chap:einleit} Der Computer ist ein Werkzeug, welches
dem Menschen Arbeit abnehmen kann. Um komplizierte Aufgaben zu übernehmen, muss
sich der Computer jedoch an menschliches Verhalten, menschliches Urteilsvermögen
und an menschliche Intelligenz annähern. Somit benötigt der Computer oder das
steuernde Computerprogramm eine künstliche Intelligenz. Die Entwicklung eines
intelligenten Computerprogrammes birgt verschiedene Herausforderung. Der
fähigste und am weitesten verbreitete Ansatz an diese Herausforderungen liefert
Machine Learning. Diese Arbeit ist eine Untersuchung Im Bereich Machine
Learning. Spezifischer befindet sich die Arbeit im Bereich von Deep
Reinforcement Learning, einem Teilgebiet von Machine Learning.
 
Die Fragestellung der Untersuchung lautet: Inwiefern kann eine künstliche
Intelligenz lernen, Strichbilder auf eine physische Weise nachzuzeichnen?

Für ein gegebenes Strichbild soll die künstliche Intelligenz (KI) erlernen, ein
möglichst gleiches Bild daneben zeichnen zu können. Der Prozess des
Nachzeichnens ist dabei durch verschiedene Kriterien definiert, die in dieser
Arbeit beschrieben werden. Die KI soll das Nachzeichnen von Strichbildern
allgemein erlernen. Strichbilder können Ziffern, Buchstaben, Formen, Symbole und
weitere einfache Zeichnungen sein. Die KI soll diese verschiedenen Motive
vergleichbar gut nachzeichnen. Das Format der Zeichnungen ist dabei auf eine
feste grösse und eine Farbtiefe von $1$ (schwarzweiss) beschränkt.
 
Nachzeichnen ist eine menschliche Tätigkeit. Menschen führen beim Zeichnen durch
gewisse Handbewegungen einen Stift, wodurch das Nachzeichnen mit physischen
Einschränkungen verbunden ist. Der Stift teleportiert sich nicht, sondern bewegt
sich mit einer limitierten Geschwindigkeit. Die KI soll das Nachzeichnen mit
ähnlichen physischen Einschränkungen erlernen. Das heisst, die KI soll lernen,
einen Stift zu führen. Die physischen Einschränkungen sind für die KI jedoch
simuliert und im Vergleich zu der echten Welt vereinfacht. Trotzdem sollte es
möglich sein, mit der KI einen zeichnenden Roboter zu steuern
 
Die Untersuchung behandelt folgende Unterfragen, welche die Fragestellung
herunterbrechen und ausweiten.
\begin{itemize}
   \item Wie kann die Architektur einer KI aussehen, die das Nachzeichnen erlernt?
   \item Wie lässt sich die Leistung der KI in ihrer Aufgabe beurteilen?
   \item Wie lässt sich die Leistung der KI in ihrer Aufgabe verbessern?
   \item Wie ändert sich die Leistung der KI für Strichbilder, die im Training nicht enthalten sind?
   \item Wie und inwiefern lässt sich das Verhalten der KI mit menschlichem Zeichnen vergleichen?
   \item Kann eine KI Strichbilder ohne Vorlage zeichnen?
\end{itemize}
