\chapter{Vorwort}
Diese Arbeit ist eine Untersuchung über künstliche Intelligenz. Die
Fragestellung der Untersuchung wird mit einer selbst programmierten künstlichen
Intelligenz beantwortet. 

Wir haben uns für das Thema Künstliche Intelligenz entschieden, weil damit
praktische Arbeit mit intellektueller Forschung verbunden werden kann. Das Thema
ermöglicht ausgeprägte Programmierarbeiten, was uns zuspricht. Zusätzlich
ermöglicht künstliche Intelligenz einfache Forschung. Mit einfacher Forschung
ist dabei nicht der Grad der Komplexität gemeint, sondern, dass die Forschung
noch nicht zu weit fortgeschritten ist. Es gibt Aspekte und Anwendungen von
künstlicher Intelligenz, die für uns zugänglich sind und noch nicht zuweit
erforscht sind, um neue Ideen zu finden. Ausserdem benötigt die Forschung an
künstlicher Intelligenz nur einen Computer. Experimente und Tests können durch
Programmcode ausgeführt werden. Die Auswertung der Ergebnisse findet auf dem
gleichen Computer statt und die Genauigkeit der Ergebnisse ist auch kein
Problem, weil diese direkt vom Computer berechnet werden. Der Computer ist eine
optimale Umgebung für eine erste Forschungsarbeit.

Diese Arbeit ist für uns eine erste vertiefte Erfahrung mit dem grossen Gebiet
der künstlichen Intelligenz. Wir erhoffen uns durch diese Erfahrung einen
erweiterten Horizont, neues Wissen und verbesserte Programmierkenntnisse.

Vielen Dank an unseren Betreuer, Nicolas Ruh, für die hilfreichen Vorschläge,
die ausgeprägte Beratung und das Vertrauen in uns.

