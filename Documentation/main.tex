%% (Master) Thesis template
% Template version used: v1.4
%
% Largely adapted from Adrian Nievergelt's template for the ADPS
% (lecture notes) project.


%% We use the memoir class because it offers a many easy to use features.
\documentclass[11pt,a4paper,titlepage]{memoir}

%% Packages
%% ========

%% LaTeX Font encoding -- DO NOT CHANGE
\usepackage[OT1]{fontenc}

%% Babel provides support for languages.  'english' uses British
%% English hyphenation and text snippets like "Figure" and
%% "Theorem". Use the option 'ngerman' if your document is in German.
%% Use 'american' for American English.  Note that if you change this,
%% the next LaTeX run may show spurious errors.  Simply run it again.
%% If they persist, remove the .aux file and try again.
\usepackage[ngerman, english]{babel}

%% Input encoding 'utf8'. In some cases you might need 'utf8x' for
%% extra symbols. Not all editors, especially on Windows, are UTF-8
%% capable, so you may want to use 'latin1' instead.
\usepackage[utf8]{inputenc}

%% This changes default fonts for both text and math mode to use Herman Zapfs
%% excellent Palatino font.  Do not change this.
\usepackage[sc]{mathpazo}

%% The AMS-LaTeX extensions for mathematical typesetting.  Do not
%% remove.
\usepackage{amsmath,amssymb,amsfonts,mathrsfs}

%% NTheorem is a reimplementation of the AMS Theorem package. This
%% will allow us to typeset theorems like examples, proofs and
%% similar.  Do not remove.
%% NOTE: Must be loaded AFTER amsmath, or the \qed placement will
%% break
\usepackage[amsmath,thmmarks]{ntheorem}

%% LaTeX' own graphics handling
\usepackage{graphicx}

%% We unfortunately need this for the Rules chapter.  Remove it
%% afterwards; or at least NEVER use its underlining features.
\usepackage{soul}

%% This allows you to add .pdf files. It is used to add the
%% declaration of originality.
\usepackage{pdfpages}

%% Some more packages that you may want to use.  Have a look at the
%% file, and consult the package docs for each.
%% See the TeXed file for more explanations

%% [OPT] Multi-rowed cells in tabulars
%\usepackage{multirow}

%% [REC] Intelligent cross reference package. This allows for nice
%% combined references that include the reference and a hint to where
%% to look for it.
\usepackage{varioref}

%% [OPT] Easily changeable quotes with \enquote{Text}
%\usepackage[german=swiss]{csquotes}

%% [REC] Format dates and time depending on locale
\usepackage{datetime}

%% [OPT] Provides a \cancel{} command to stroke through mathematics.
%\usepackage{cancel}

%% [NEED] This allows for additional typesetting tools in mathmode.
%% See its excellent documentation.
\usepackage{mathtools}

%% [ADV] Conditional commands
%\usepackage{ifthen}

%% [OPT] Manual large braces or other delimiters.
%\usepackage{bigdelim, bigstrut}

%% [REC] Alternate vector arrows. Use the command \vv{} to get scaled
%% vector arrows.
\usepackage[h]{esvect}

%% [NEED] Some extensions to tabulars and array environments.
\usepackage{array}

%% [OPT] Postscript support via pstricks graphics package. Very
%% diverse applications.
%\usepackage{pstricks,pst-all}

%% [?] This seems to allow us to define some additional counters.
%\usepackage{etex}

%% [ADV] XY-Pic to typeset some matrix-style graphics
%\usepackage[all]{xy}

%% [OPT] This is needed to generate an index at the end of the
%% document.
%\usepackage{makeidx}

%% [OPT] Fancy package for source code listings.  The template text
%% needs it for some LaTeX snippets; remove/adapt the \lstset when you
%% remove the template content.
\usepackage{listings}
\lstset{
    %language=C++, % We have multiple languages in our thesis
    basicstyle={\normalfont\ttfamily},
    frame=tb,
    framexleftmargin=1.75em,
    xleftmargin=1.75em,
    %xrightmargin=0.25em,
    numbers=left,
    stepnumber=1,
    showstringspaces=false,
    breaklines=true
}

\lstdefinestyle{C++}{
    language=C++,
    keywordstyle=\color{blue},
    stringstyle=\color{red},
    commentstyle=\color{green},
    morecomment=[l][\color{magenta}]{\#}
}

%% [REC] Fancy character protrusion.  Must be loaded after all fonts.
\usepackage{microtype}

%% [REC] Nicer tables.  Read the excellent documentation.
\usepackage{booktabs}

%% \usepackage[a4paper]{geometry}
%% \usepackage{caption}
 \usepackage{subcaption}
%% \usepackage{xcolor}

%% Uncomment for floatbarrier in sections
%\usepackage[section]{placeins}

%% Plots
\usepackage{pgfplots}
%\pgfplotsset{compat=1.8}
%\usepgfplotslibrary{statistics}

%% Filetree
\usepackage{dirtree}

%% Draw diagram
\usetikzlibrary{arrows,positioning,shapes}

%% Disable word-break
%\usepackage[none]{hyphenat}
\hyphenpenalty 10000

%% List spacing
\usepackage{enumitem}
\setlist{nosep}
%\setlist{noitemsep}

%% Our layout configuration.  DO NOT CHANGE.
\input{layout/layoutsetup}

%% Theorem environments.  You will have to adapt this for a German
%% thesis.
\input{layout/theoremsetup}

%% Helpful macros.
\input{layout/macrosetup}

%% Make document internal hyperlinks wherever possible. (TOC, references)
%% This MUST be loaded after varioref, which is loaded in 'extrapackages'
%% above.  We just load it last to be safe.
\usepackage[linkcolor=black,colorlinks=true,citecolor=black,filecolor=black]{hyperref}


% Our packages
% Biblatex for a better bibliography
\usepackage{hyphenat}
\usepackage{array}
\usepackage{caption}
\usepackage{subcaption}
\usepackage{makecell}
% \usepackage[citestyle=numeric, bibstyle=numeric, sorting=none]{biblatex}
\usepackage[backend=biber, style=apa, 
natbib=true]{biblatex}
\usepackage{cleveref}
\usepackage[babel, german=quotes]{csquotes}
\addbibresource{refs.bib}

% Our configs
\emergencystretch=\maxdimen
% \DeclareLabeldate{
%   \field{date}
%   \field{event}
%   \field{orig}
%   \field{url}
%   \literal{nodate}
% }

\renewcommand\cite{\citep}

\selectlanguage{ngerman}
%% Document information
%% ====================

\title{ReSketch AI}
% \author{Ian Wasser, Robin Steiner}
\thesistype{Maturarbeit}
% \advisors{Betreut durch: Nicolas Ruh \\ Zweitbeurteilung: Dieter Koch}
\date{11. Oktober 2022}

\begin{document}

\newcommand*{\doubleref}[1]{\hyperref[{#1}]{\ref*{#1} \nameref*{#1}}}
%https://tex.stackexchange.com/questions/121865/nameref-how-to-display-section-name-and-its-number

\frontmatter

%% Title page is autogenerated from document information above.  DO
%% NOT CHANGE.
\begin{titlingpage}
  \calccentering{\unitlength}
  \begin{adjustwidth*}{\unitlength-24pt}{-\unitlength-24pt}
    \maketitle
  \end{adjustwidth*}
\end{titlingpage}

%% The abstract of your thesis.  Edit the file as needed.
\selectlanguage{english}
\begin{abstract}\label{abstract} ReSketch ist eine künstliche Intelligenz, die
Strichbilder nachzeichnen kann. Strichbilder sind beispielsweise Ziffern oder
Buchstaben. Die künstliche Intelligenz kann sich beim Zeichnen so bewegen, wie
es mit einem echten Stift möglich wäre. ReSketch funktioniert mit Deep
Q-Learning, einem Reinforcement Learning Modell. Das Modell basiert dabei auf
der Arbeit hinter Doodle-SDQ \cite{zhou_learning_2018}, erfährt aber
verschiedene Erweiterungen. Die Leistung von dem Modell wird durch vordefinierte
Kriterien evaluiert, deren Werte das Resultat dieser Arbeit ausmachen. ReSketch
erreicht eine Übereinstimmung von $90\%$ zwischen der Vorlage und dem
nachgezeichneten Bild. Ausserdem kann die KI nach dem Training beliebige Arten
von Strichbildern nachzeichnen, obwohl diese lediglich auf das Zeichnen von
Zahlen trainiert ist. Eine zweite künstliche Intelligenz, die auf der
nachzeichnenden KI basiert, entfernt sich von der ursprünglichen Aufgabe. Diese
zweite KI erlernt das selbstständige Zeichnen von einem ausgewählten Motiv, ohne
eine Vorlage davon zu erhalten. Zu diesem Zweck werden die generierten
Zeichnungen der KI mit einem Klassifizierungsmodell bewertet. Mit einem
spezifischen Training dieser generativen KI können verschiedene Handschriften
emuliert werden.

\end{abstract}

    
\newpage
    
\section*{Vorwort}\label{vorwort} Diese Arbeit ist eine Untersuchung im Bereich
der künstlichen Intelligenz. Die Fragestellung der Untersuchung wird mithilfe
einer selbst programmierten künstlichen Intelligenz beantwortet.
    
Wir haben uns für das Thema künstliche Intelligenz entschieden, weil dabei
praktische Arbeit mit intellektueller Forschung verbunden wird. Das Thema
ermöglicht ausgeprägte, praktische Programmierarbeiten, was uns zuspricht.
Zusätzlich ermöglicht künstliche Intelligenz einfache Forschung. Mit einfacher
Forschung ist dabei nicht der Grad der Komplexität gemeint, sondern die Vielfalt
der Möglichkeiten. Es gibt Aspekte und Anwendungen der künstlichen Intelligenz,
die für Schüler zugänglich sind und noch nicht zu weit erforscht sind, um neue
Ideen zu finden. Ausserdem benötigt die Forschung an künstlicher Intelligenz nur
einen Computer. Experimente und Tests können durch Programmcode ausgeführt
werden. Die Auswertung der Experimente findet auf demselben Computer statt und
die Genauigkeit der Ergebnisse stellt ebenfalls kein Problem dar, da der
Computer die Zahlen direkt berechnet. Der Computer ist eine optimale Umgebung
für eine erste Forschungsarbeit.
    
Diese Arbeit ist für uns eine erste vertiefte Erfahrung mit dem Gebiet der
künstlichen Intelligenz. Wir erhoffen uns durch diese Erfahrung einen
erweiterten Horizont, neues Wissen und verbesserte Programmierkenntnisse.

Der gesamte Code der künstlichen Intelligenz mit all ihren Komponenten und Versionen ist in dem folgenden
GitHub Repository einsehbar:
\url{https://github.com/LarsZauberer/Nachzeichner-KI}.

    
Vielen Dank an unseren Betreuer der Maturarbeit, Dr.\ Nicolas Ruh und an unseren
Experten im Rahmen des nationalen Wettbewerbs von Schweizer Jugend Forscht, Dr.\
Michael Tschannen, für die hilfreichen Vorschläge, die ausgeprägte Beratung und
das an uns weitergegebene technische Wissen. Vielen Dank auch an Dieter Koch für
die Zweitbeurteilung und an Günther Wasser für das Korrekturlesen dieser Arbeit.


\selectlanguage{ngerman}

%% TOC with the proper setup, do not change.
\cleartorecto
\begin{KeepFromToc}
\tableofcontents
\end{KeepFromToc}
\mainmatter

%% Template content
%\input{template/introduction}
%\input{template/rules}
%\input{template/sections}
%\input{template/typography}

%% Your real content!
\chapter{Einleitung}\label{chap:einleit}
Der Computer ist ein Werkzeug, das dem Menschen Arbeit abnehmen kann. Um
komplizierte Aufgaben zu übernehmen, muss sich der Computer jedoch an
menschliches Verhalten, menschliches Urteilsvermögen und menschliche Intelligenz
annähern. Mit anderen Worten braucht der Computer, oder das steuernde
Computerprogramm, eine künstliche Intelligenz. Ein Intelligentes
Computerprogramm zu entwickeln ist komplex. Der fähigste und am weitesten
verbreitete Ansatz liefert Machine Learning. Diese Arbeit selbst ist eine
Untersuchung Im Bereich Machine Learning. Spezifischer ist die Arbeit im Bereich
Reinforcement Learning, einem Teilgebiet von Machine Learning.

Die Fragestellung der Untersuchung lautet: Kann eine künstliche Intelligenz
lernen, Strichbilder auf eine physische Weise nachzuzeichnen, sodass diese durch
ein automatisches System richtig erkannt werden können?

Für ein gegebenes Strichbild soll die künstliche Intelligenz (KI) erlernen, ein
möglichst gleiches Bild daneben zeichnen können. Die Frage ist, ob die KI das
Nachzeichnen genug gut lernen kann, damit die Zeichnung von einem automatischen
System richtig erkannt wird. Richtig erkannt heisst in diesem Fall vereinfacht,
dass eine zweite KI in der Zeichnung das selbe Motiv wie in der Vorlage erkennt.
Wenn das zutrifft, kann die künstliche Intelligenz erfolgreich nachzeichnen. Es
existieren allerdings weitere Kriterien, die die Leistung der KI bei der
Tätigkeit des Nachzeichnens beurteilen.

Nachzeichnen ist eine menschliche Tätigkeit. Menschen führen beim Zeichnen durch
gewisse Handbewegungen einen Stift, wodurch das Nachzeichnen mit physischen
Einschränkungen verbunden ist. Der Stift kann sich nicht teleportieren, sondern
sich nur mit einer bestimmten Geschwindigkeit fortbewegen. Die KI soll das
Nachzeichnen mit ähnlichen physischen Einschränkungen erlernen. Mit anderen
Worten soll die KI lernen, einen Stift zu führen.  Die
physischen Einschränkungen sind dabei jedoch simuliert und im Vergleich zu der
echten Welt vereinfacht. 

Die KI soll das Nachzeichnen von Strichbildern allgemein erlernen. Strichbilder
können Zahlen, Buchstaben, Formen, Symbole und allgemeine Kritzeleien sein.
Natürlich kann die KI nicht mit allen Arten von Strichbildern trainiert werden,
weil die Vielfalt zu gross ist. Daraus ergibt sich die Frage, wie gut die
künstliche Intelligenz Arten von Strichbildern nachzeichnet, die nicht im
Training enthalten waren.

Die vorangehenden Überlegungen sind in einer Sammlung an Unterfragen, die in
dieser Arbeit beantwortet werden, vertreten. Die Unterfragen lauten:
\begin{itemize}
    \item Wie kann die Architektur einer KI aussehen, die das Nachzeichnen erlernt?
    \item Nach welchen Kriterien lässt sich die Leistung der KI in dieser Aufgabe beurteilen?
    \item Wie lässt sich die Leistung der KI in dieser Aufgabe verbessern?
    \item Wie ändert sich die Leistung der KI für Strichbilder, die im Training nicht enthalten sind?
    \item Welche Einflüsse haben physische Einschränkungen auf die Leistungs der KI?
    \item Wie und in wiefern lässt sich die KI mit menschlichem Zeichnen vergleichen?
\end{itemize}



\chapter{Theoretische Grundlagen}

\section{Machine Learning}
\label{chap:t_ml}
Teilbereich künstliche Intelligenz (Künstliche Intelligenz ist Maschine, die
menschliche Tätigkeiten nachmacht) Maschine = Computer -> Computerprogramm
Machine Learning ist Überbegriff -> Computerprogramme, die eine Mustererkennung
entwickeln. Durch Analysieren von Daten. KI lernt Vorhersagen

Beispiel Zahlenerkennung: Beispiel für Künstliche Intelligenz Bild einer
handschriftlichen Zahl wird Programm übergeben (Input) Das Programm bestimmt, um
welche Zahl es sich handelt. Programm gibt Zahl aus

Künstliche Intelligenz weil das Erkennen von Zahlen menschlich ist.

Unterschied machine Learning: KI lernt von Grund auf, Zahlen zu Erkennen Anfang
wird allermeistens Falsch sein

KI lernt durch die Analyse von Daten. -> Trainingsdaten. Dabei ist bekannt, was
die richtige Lösung zu den gegebenen Daten ist. So erfährt die KI, wenn sie
falsch liegt. Programm kann sich bezogen auf Erfahrung selbst anpassen.
Anpassungen möglichst so, dass Vorhersagen in Zukunft besser sind.

Nachdem KI mit Trainingsdaten verbessert wurde, sollte diese auch für neue Daten
möglichst genaue Vorhersagen machen.

Nächste Frage: Wie trifft KI entscheidungen, und wie kann sie sich selbst
Anpassen? -> Antwort Neuronale Netze


\subsection*{künstliche neuronale Netze}

Grundbaustein eines neuronalen Netzes sind  Neuronen, die mehrere Eingaben
erhalten. Jede Eingabe wird verschieden gewichtet, wobei das Gewicht mit der
Eingabe (0 oder 1) multipliziert wird. (vereinfachte Erklärung eines
Sigmoid-Neuron)

Wenn Alle Eingaben + Gewichte Addiert werden und den Threshold, einen
vorgegebenen Grenzwert, überschreiten, ist der Output dieses Neurons = 1. -> Das
Neuron feuert

Neuronales Netz = verbindung von künstlichen Neuronen. Künstliche Neuronen
werden von mehreren Eingaben beeinflusst. Die Eingaben haben dabei eine
Unterschiedliche Gewichtung. Das künstliche Neuron hat nur eine Ausgabe zwischen
0 und 1.

In einem neuronalen Netz werden viele Neuronen miteinander verbunden. Der
Eingabe entsprechen dabei die Daten, die dem neuronalen Netz übergeben werden.
Danach dient die Ausgabe von Neuronen als die Eingabe von weiteren Neuronen.
Umso grösser die Ausgabe eines Neurons ist, desto stärker beeinflusst dieses die
Ausgabe der Neuronen, an deren Eingaben es beteiligt ist.

Die Entscheidung des Programms wird auch in verschiedenen Neuronen dargestellt.
Im Beispiel sind es 10 Neuronen, wobei jedes Neuron eine andere Zahl von 0-9
darstellt. Diese Neuronen haben keine Ausgabe mehr, die für das neuronale Netz
relevant ist. Die Entscheidung basiert in diesem Fall lediglich darauf, welches
der 10 Ausgabeneurenen den höchsten Wert enthält. Die Zahl, die dieses Neuron
repräsentiert, wird dann als die Entscheidung des Neuronalen Netzes angesehen

Der Lernprozess findet durch eine anpassung von Einzelnen Gewichten der
Eingaben, wenn die KI eine falsche Entscheidung trifft. Dadurch soll die
Entscheidung so angepasst werden, dass sie im nächsten Fall besser ausfällt

Neuronale Netze sind in Ebenen aufgeteilt. Eingabe Ebene, Ausgabe Ebene. Und
Ebenen dazwischen. Versteckte Ebenen. Sobald es mehr als eine Ebene an
versteckten Ebene gibt, nennt man es Deep learning.

(Convolutional Layers?)




\section{Reinforcement Learning}
\label{chap:t_rl}
Verschiedene Arten


\section{Verwandte Arbeiten und Themen}
\label{chap:t_verwandt}

Ein verwandtes Thema dieser Arbeit ist die Robotik. Ein Roboter ist laut der
Definition eine `Apparatur, die bestimmte Funktionen eines Menschen ausführen
kann' (Duden). Das Nachzeichnen von Strichbildern ist ebenfalls eine menschliche
Tätigkeit. Diese Arbeit untersucht allerdings nur das Computerprogramm, welches
diese Tätigkeit verrichten kann. Die Apparatur, die von diesem Programm
gesteuert werden könnte, wird also nicht erbaut.

Es gibt verschiedene Ansätze, um ein Computerprogramm die menschliche Tätigkeit
des Nachzeichnens verrichten zu lassen. Ein häufiger Ansatz ist "Stroke-Based
Rendering", wobei Bilder durch das Platzieren von Elementen wie Strichen
gezeichnet werden. Beispiele für Arbeiten, die diesen Ansatz verwenden sind\dots
Stroke-Based Rendering unterscheidet sich von menschlichem Zeichnen dadurch,
dass kein Stift geführt wird. Stattdessen können die Elemente zu jedem Zeitpunkt
an einer willkürlichen Position auf der Zeichenfläche platziert werden.

Andere Ansätze simulieren die Führung eines Stiftes. Damit ist gemeint, dass das
Computerprogramm nicht zu jedem Zeitpunkt an jedem Ort Zeichnen kann.
Stattdessen muss das Programm einen virtuellen Stift bewegen und kann nur gerade
dort zeichnen, wo sich der Stift befindet. Das ist eine Einschränkung, die auch
auf menschliches Zeichnen zutrifft. Ein Beipiel für diese Art des Zeichnens ist
Doodle-SDQ. Das Computerprogramm dieser Arbeit basiert auf dem Programm von
Doodle-SDQ 

\subsection*{Doodle-SDQ}
Doodle-SDQ ist ein Programm, das durch Deep Q Learning erlernt hat, Strichbilder
aus dem Google Quick-Draw Datenset nachzuzeichnen. Ist in der Freiheit des
Zeichnens eingeschränk\indent 

Architektur: Agent = Stift. Umgebung = Zeichenfläche. Speichert Position von
Agent, das zu Nachzeichnende Bild, das was bis jezt gezeichnet wurde und ob
Stift gerade vom Blatt gehoben ist. Die Belohnung/Bestrafung ist der Grad der
Ähnlichkeit zwischen Bildern. Die Unmittelbare Umgebung des Agenten, in dem er
sich in einem Schritt bewegen kann, wird in das neuronale Netz noch ein weiteres
Mal eingegeben, wodurch darauf ein Fokus gelegt wird. Ausgabe des neuronalen
Netz ist eine der Aktionen die der Agent ausführen kann
R

\section{Git und GitHub}
\label{chap:t_git}




\chapter{Methode}\label{chap:m} Die Methode dieser Untersuchung besteht darin,
die in der Fragestellung beschriebene künstliche Intelligenz (KI) zu entwickeln
und dessen Leistung auszuwerten (siehe \doubleref{chap:m_auswert}). Die
Diskussion dieser Resultate führt schlussendlich zu einer Antwort auf die
Fragestellung. Die Entwicklung der KI besteht aus zwei Teilen. Der eine Teil
umfasst die Definition der Kriterien, nach denen die Leistung der KI evaluiert
wird (siehe \doubleref{chap:m_eval}). Der andere Teil umfasst die Entwicklung
der KI. Dazu gehört die Entwicklung einer grundlegenden Architektur (siehe
\doubleref{chap:m_grund}), sowie die Vollendung der KI mit verschiedenen
Variationen und Ansätzen (siehe \doubleref{chap:m_var}). Die Variationen sind
dabei Versuche, die Leistung der KI zu maximieren oder ihr Verhalten zu
verändern. Ein spezielle Variation, die sich dabei von der ursprünglichen
Absicht entfernt, ist die Umwandlung in eine generative KI, die nicht mehr
Nachzechnet und stattdessen eigene Zeichnungen ohne Vorlage kreiert (siehe
\doubleref{chap:m_generativ}).


\section{Grundprogramm}\label{chap:m_grund} Das Grundprogramm ist eine flexibel
anwendbare Architektur der KI, die als Grundlage für eine Vielzahl an
Variationen dient. Das Grundprogramm bietet dabei eine Trainingsumgebung für die
KI, eine Zeichenumgebung und das Reinforcement Learning Modell (Deep
Q-Learning), zusammen mit dem Agent und dem neuronalen Netz (siehe
\doubleref{sub:t_rl_func}). Das Grundprogramm beinhaltet dabei keine
Reward-Function und ist somit keine funktionale Version der KI.
Das Grundprogramm ist in Python unter der Verwendung des Keras
Frameworks implementiert (siehe \doubleref{chap:t_ml}).


\subsection{Doodle-SDQ als Basis}\label{sub:m_grund_dood} Das Reinforcement
Learning Modell des Grundprogramms basiert auf Doodle-SDQ (siehe
\doubleref{sub:t_ver_dood}). Von Doodle-SDQ ist das neuronale Netz, bezogen auf
die Form des Inputs, des Outputs und den Hidden Layers, grösstenteils
übernommen. Die relevanten Anpassungen zwischen Doodle-SDQ und dem Grundprogramm
dieser Arbeit sind nachfolgend erläutert.
 
Bei der Umgebung handelt es sich, wie bei Doodle-SDQ, um eine Zeichenfläche,
worauf sich der Agent bewegt. Das Grundprogramm wird auf das
Nachzeichnen von Ziffern trainiert. Die Ziffern stammen aus dem MNIST Datenset
(siehe \doubleref{chap:t_ml}) und haben somit eine Grösse von $28\times28$
Pixeln. Die Fläche, worauf sich der Agent bewegen  
kann, hat folglich auch eine Grösse von $28\times28$ Pixeln. Der Global Stream
(siehe \doubleref{sub:t_ver_dood}) des Inputs in das neuronale Netz ändert sich
bis auf die neue Grösse der Bilder nicht. Die Pixel der Bilder, wie auch die
Zeichenfläche, nehmen den Wert von einem Bit an. Eine Null repräsentiert einen
schwarzen (nicht gezeichneten) Pixel an dieser Stelle im Bild und eine Eins
einen weissen (gezeichneten) Pixel. Die genaue Architektur des neuronalen Netzes
ist in der folgenden Abbildung angegeben (siehe \autoref{fig:architecture}). 
 
%Bild architecture
\begin{figure}[!ht]
 \centering
 \includegraphics[width=\textwidth-4cm]{images/methode/architecture.png}
 \caption{Architektur des neuronalen Netzes im Grundprogramm (eigene Abbildung, mit Keras erstellt). Jeder Block repräsentiert einen Layer. Die Form des Inputs und des Outputs ist von jedem Layer angegeben}\label{fig:architecture}
\end{figure}
 
Der Local Stream und damit auch der Local image Patch schrumpfen von $11\times11$
Pixel auf $7\times7$ Pixel. Somit schrumpft gleichzeitig der Action-Space (siehe
\doubleref{sub:t_rl_func}) des Agenten von $2\cdot11\cdot11 = 242$ Actions auf
$2\cdot7\cdot7 = 98$ Actions. Das bedeutet für den Agent, dass er sich pro Step
um maximal drei Pixel von seiner Position wegbewegen kann. Diese Bewegung kann
der Agent entweder zeichnend oder nicht zeichnend ausführen (siehe
\autoref{fig:actionspace}).
 
%bild normal actionspace
\begin{figure}[!ht]
 \centering
 \includegraphics[width=\textwidth]{images/methode/actionspace.png}
 \caption{Action-Space im Grundprogramm}\label{fig:actionspace}
\end{figure}
 
Falls der Agent die Action zeichnend ausführt, zieht das Programm einen Strich
zwischen der alten und der neuen Position. Das bedeutet, dass alle Pixel
der Zeichenfläche zwischen den beiden Positionen weiss werden. Der Strich hat eine
festgelegte Breite von $3$ Pixeln. Am Anfang jeder Episode, also mit jeder neuen
Ziffer, startet der Agent in einer zufälligen Position im nicht zeichnenden
Zustand. Am Anfang jeder Episode ist die Zeichenfläche leer, also vollkommen
Schwarz.


\subsection{Erweiterungen}\label{sub:m_grund_erweiterungen} Die folgenden
Eigenschaften des Grundprogramms sind Erweiterungen der Architektur von
Doodle-SDQ.

Die erste Erweiterung beschreibt die behandlung von illegalen Actions. Alle
Actions, die den Agent über die vorgegebene Zeichenfläche hinaus positionieren
würden, sind nicht zulässig. Diese Actions sind für den Agent nicht wählbar
und ihr optimaler Q-Value (siehe \doubleref{sub:t_rl_func}) ist in jedem Fall
$0$. Das hat zur Folge, dass nach dem Training die allermeisten unzulässigen
Actions einen Q-Value nahe oder gleich $0$ haben. Das senkt die
Wahrscheinlichkeit, dass der Agent versucht, eine unzulässige Action
auszuführen.

Die zweite Erweiterung betrifft den Action-Space mit der Einführung einer Stopp
Action. Wenn der Agent diese Action wählt, bricht die Zeichnung ab. Der Agent
kann somit frei entscheiden, wann die Zeichnung fertig ist. Die Stopp Action
kann während dem Training dem Agent verboten werden. In diesem Fall wird die
Action als unzulässig behandelt.

 
\subsection{Präparierung der Daten und Optimierung}\label{sub:m_grund_data}
Die Trainingsdaten bestehen aus $36'000$ Bildern von handgeschriebenen Ziffern
aus dem MNIST Datenset (siehe \doubleref{chap:t_ml}). Die restlichen Bilder des
MNIST Datensets machen die Testdaten aus. Die Bilder im Datenset sind als Bitmap
dargestellt, wobei jedes Element (jeder Pixel) einen Wert zwischen $0$ und $255$
annimmt. Die Zahl repräsentiert eine Graustufe, wobei $0$ Schwarz ist und $255$
Weiss. Diese Graustufen werden entfernt. Jeder Pixel mit einem Wert über $0$
übernimmt den Wert $1$, wodurch die Bilder nur noch aus Einsen und Nullen
bestehen. Dabei ist $0$ Schwarz und $1$ Weiss (siehe \autoref{fig:norm-v-nogray}).
So stimmen die Bilder mit den Zeichnungen, die der Agent produzieren kann,
überein.
 
%Bild normal num vs nogray num
\begin{figure}[!ht]
 \centering
 \includegraphics[width=\textwidth]{images/methode/norm-v-nogray.png}
 \caption{Entfernung der Graustufen im MNIST Datenset. (eigene Abbildung)}\label{fig:norm-v-nogray}
\end{figure}
 
 
Das Grundprogramm trainiert mit $5'000$ Bildern, von denen jede Ziffer $500$
Bilder ausmacht. Die restlichen Bilder in den Trainingsdaten sind für mögliche 
Erweiterungen aufgehoben. Der Agent zeichnet jedes der $5'000$ Bilder ein Mal und
trainiert somit für $5'000$ Episodes. Der Agent zeichnet für $64$ Steps pro
Episode, falls die Stopp Action nicht früher gewählt wird.
 
Die Hyperparameter aller Versionen der KI sind durch den Bayesian Optimization
Algorithmus optimiert (siehe \doubleref{sub:t_ml_hyper}). Die Implementierung
des Algorithmus in Python stammt von \cite{fernando_nogueira_bayesian_2014}. Der
Algorithmus ändert sich für verschiedene Variationen der KI nicht und ist somit
Teil des Grundprogramms. Mit jeder Iteration des Baysian Optimization
Algorithmus trainiert das Reinforcement Learning Modell für eine vom Algorithmus
selbst bestimmte Anzahl Episodes. Die Zielvariable, die durch den Baysian
Optimization Algorithmus maximiert werden soll, wird am Ende jeder Iteration des
Trainings in der Testumgebung berechnet (siehe \doubleref{sub:m_auswert_test}).
Unter welchem Kriterium (siehe \doubleref{chap:m_eval}) der Wert der
Zielvariable berechnet wird ist frei wählbar  
basiert, ist frei wählbar.
 
 
\section{Evaluation der Leistung}\label{chap:m_eval}
In diesem Unterkapitel sind die Kriterien definiert, welche die Leistung der
künstlichen Intelligenz evaluieren. Spezifischer beschreiben die
Kriterien, wie gut die KI nachzeichnet. Für eine präzise und objektive
Evaluation sind alle Kriterien durch einen Zahlenwert repräsentiert. Die
Kriterien und ihre jeweilige Berechnung werden nachfolgend beschrieben.
 
\subsection{Erkennbarkeit}\label{sub:m_eval_rec}
Das Kriterium der Erkennbarkeit beschreibt, ob in der Vorlage das gleiche Motiv
wie in der Zeichnung der künstlichen Intelligenz erkannt wird. Wenn
Beispielsweise in beiden Fällen eine Fünf erkannt wird, hat das Kriterium den
Wert $1$. Wird in der Vorlage eine Fünf erkannt, aber in der Zeichnung eine
Vier, hat das Kriterium den Wert $0$
 
Das erkannte Motiv wird durch eine zweite KI beurteilt (siehe
\doubleref{sub:t_ml_func}). Diese klassifizierden Machine Learning Modelle sind
spezifisch auf Bilder ohne Graustufen trainiert, welche die nachzeichnende KI
dieser Arbeit produzieren kann. Der Output Layer der klassifiziernden KI hat
Softmax mit einer hohen Temperatur als Activation Function (siehe
\doubleref{chap:t_ml}). Das versichert, dass die KI nur eine eindeutige
Klassifizierung trifft wenn die Wahrscheinlichkeit für dessen Richtigkeit hoch
ist. Für die verschiedenen Arten von Strichbildern, welche die KI zeichnen soll,
werden spezifische klassifizierende Machine Learning Modelle mit den zugehörigen
Datensets trainiert. Diese Modelle, zusammen mit ihrer Genauigkeit, sind in der
folgenden Tabelle (siehe \autoref{tab:models}) aufgeführt. Das Neuronale Netz
der Modelle stammt aus einem online Machine Learning Kurs \cite{wang_deep_2021}. %Todo Quelle Online Modelle

\begin{table}[!ht]
 \centering  
 \begin{tabular}{|l|l|l|l|}
 \hline
     Art & Trainiert mit & Genauigkeit [\%]\\ \hline
     Ziffern & MNIST & 99\\ \hline
     Buchstaben  & EMNIST Letters & 91\\ \hline
     \makecell{Strichbilder\\von Objekten}  & Auswahl aus QuickDraw & 98\\ \hline
 \end{tabular}
 \caption{Vortrainierte Modelle}\label{tab:models}
\end{table}
 
\subsection{Prozentuale Übereinstimmung}\label{sub:m_eval_proc}
Dieses Kriterium beschreibt die prozentuale Übereinstimmung der weissen
(gezeichneten) Pixel zwischen der Vorlage und der Zeichnung der KI (siehe
\doubleref{chap:m_grund}). Der Wert $K$ dieses Kriteriums zu dem Step $t$
berechnet sich aus folgender Formel:
\[ K(t) = \frac{G(t)}{G_{\max}} \] $G_{\max}$ entspricht der Anzahl aller
weissen Pixeln in der Vorlage. $G(t)$ entspricht der Anzahl der weissen Pixel,
die zwischen der Vorlage und der Zeichenfläche übereinstimmen. Die Pixel, die
nicht übereinstimmen, zählen negativ für $G(t)$. $G(t)$ und somit auch $K(t)$
können dadurch auch negative Werte annehmen. Der maximale Wert von $K(t)$ ist 1,
was einer prozentualen Übereinstimmung von $100\%$ entspricht (siehe
\autoref{fig:ubereinstimmung}).
 
%bild übereinstimmung
\begin{figure}[!ht]
 \centering
 \includegraphics[width=\textwidth]{images/methode/ubereinstimm.png}
 \caption{Drei Beispiele für den Wert des Kriteriums der Übereinstimmung. (eigene Abbildung)}\label{fig:ubereinstimmung}
\end{figure}
 
\subsection{Geschwindigkeit}\label{sub:m_eval_speed} Dieses Kriterium
beschreibt, wie schnell die Zeichnung der KI fertig ist. Der Wert des Kriteriums
entspricht dabei der Anzahl Steps bis zur Fertigstellung. Der Punkt der
Fertigstellung ist dabei bei der letzten zeichnenden Action des Agents
definiert. Falls der Agent sich also nach dem Zeichnen weiterhin
nicht-Zeichnende Actions ausführt, so werden diese nicht mehr zu der Zeichnung
gezählt, weil die Zeichnung nicht beeinflusst wird.

\subsection{Zeichnende Zeit}\label{sub:m_eval_zeichnend} Das Kriterium der
zeichnenden Zeit beschreibt, in wie vielen Steps pro Episode die KI eine
zeichnende Action ausführt. Dieser Wert ist als ein prozentualer Anteil
angegeben und berechnet sich somit aus der Anzahl zeichnender Steps dividiert
durch die Anzahl aller Steps pro Zeichnung.

\subsection{Übermalung}\label{sub:m_eval_uebermalung} Das Kriterium der
Übermalung beschreibt, wie viele Pixel pro zeichnung mehrfach bemalt werden.
Jeder Pixel, auf dem die KI also zwei oder mehr Mal malt, erhöht den Wert dieses
Kriteriums um Eins. Jeder Pixel trägt dabei höchstens einmal zu diesem Kriterium
bei. Das heisst, dass vielfach bemalte Pixel gleich behandelt werden wie
zweimal bemalte Pixel. 


 
\section{Variationen}\label{chap:m_var}
Dieses Unterkapitel beschreibt verschiedene Variationen, ausgehend vom
Grundprogramm (siehe \doubleref{chap:m_grund}). Bei einigen dieser Variationen
handelt es sich um konkrete Implementierungen der definierten Kriterien in die
Reward-Function (siehe \doubleref{sub:t_rl_func}). Die Reward-Function kann
dabei auf mehreren Kriterien gleichzeitig basieren. Der Unterschied zwischen den
Variationen liegt im Fokus auf die Kriterien. Einige Variationen sind
untereinander kombinierbar, andere Variationen führen strukturelle Veränderungen
für die KI ein, die über die Reward-Function hinaus gehen.
 
\subsection{Basis Reward-Function}\label{sub:m_var_base}
Die Basis Reward-Function ist die einfachste Erweiterung des Grundprogrammes
(siehe \doubleref{chap:m_grund}). Diese Reward-Function implementiert das
Kriterium der prozentualen Übereinstimmung (siehe \doubleref{sub:m_eval_proc}).
Der Reward für eine Action berechnet sich aus der Differenz zwischen der
prozentualen Übereinstimmung vor dem Ausführen der Action, und der prozentualen
Übereinstimmung nach dem Ausführen der Action (also $K(t-1)$ und $K(t)$). Somit
wird der Reward $R$ zum Step $t$ durch folgende Formel berechnet.
\[ R(t) = K(t) - K(t-1) \] Der Reward eines Steps entspricht folglich nicht der
gesamten prozentualen Übereinstimmung zu einem Step. Stattdessen Entspricht der
Reward der Veränderung der prozentualen Übereinstimmung, ausgelöst durch die
Action in einem Step. Der akkumulierte Reward (siehe \doubleref{sub:t_rl_func})
enstspricht dem absoluten Wert der prozentualen Übereinstimmung.
 
 
\subsection{Training auf Geschwindigkeit}\label{sub:m_var_speed}
Das Kriterium der Geschwindigkeit (siehe \doubleref{sub:m_eval_speed}) kann in
die Reward-Function integriert werden. Dadurch trainiert die künstliche
Intelligenz auf eine minimale Zeit bis zur Fertigstellung der Zeichnung. Die
Variation verwendet grundsätzlich die Basis Reward-Function (siehe
\doubleref{sub:m_var_base}). Die vorgeschlagene Anpassung davon sieht
folgendermassen aus: Am Ende jeder Episode wird der Reward jedes Steps in der
Episode mit einem Faktor $f$ multipliziert. Dieser Faktor berechnet sich aus
folgender Formel:
\[ f = 2 - \frac{S}{S_{\max}} \]
$S_{\max}$ entspricht der Anzahl Steps, die der Agent in einer Episode begeht
(siehe \doubleref{sub:m_grund_data}). $S$ entspricht der Anzahl Steps bis zur
Fertigstellung der Zeichnung. Der Faktor nimmt einen Wert zwischen $1$ und $2$
an. Ein grosser Faktor $f$ entspricht einem hohen Reward und wird durch eine
schnelle Fertigstellung der Zeichnung ausgelöst. Wenn der Agent die Zeichnung
bis zum Ende einer Episode nicht fertigstellt, ist $f = 1$. In diesem Fall unterscheidet sich die
Reward-Function nicht von der Basis Reward-Function. Wenn die Zeichnung früher
fertiggestellt wird, zeichnet der Agent trotzdem alle $S_{\max}$ Steps. Das
verhindert eine ungleichmässige Verteilung zwischen verschiedenen Episodes im
Replay-Buffer (siehe \doubleref{sub:t_rl_func}). In diesem Fall wird $S$
allerdings nur in dem Step gespeichert, in dem die Zeichnung zum ersten Mal die
Bedingung einer Fertigstellung erfüllt.  
 
Indem das Kriterium der Geschwindigkeit während dem Training angepasst wird,
verstärkt sich der Fokus auf eine möglichst schnelle Fertigstellung. Die
minimale prozentuale Übereinstimmung einer fertigen Zeichnung ist als $75\%$
definiert. Zu Beginn des Trainings wird dieser Wert auf $25\%$ heruntergesetzt,
und über das Training hinweg linear bis auf $75\%$ erhöht. Dadurch löst die
Reward-Function bei einer unfertigen Zeichnung bereits positive Rewards für die
Geschwindigkeit aus. Für diese Anpassung wird die Bedingung der korrekten
Erkennung (siehe \doubleref{sub:m_eval_speed}) nicht beachtet.
 
 
\subsection{Training auf Erkennbarkeit}\label{sub:m_var_rec}
Das Kriterium der Erkennbarkeit kann, anders als die anderen Kriterien, nur
teilweise in die Reward-Function integriert werden. Das Kriterium strebt eine
Erkennbarkeit, unabhängig von der Art der Strichbilder, an (siehe
\doubleref{sub:m_eval_rec}). Die künstliche Intelligenz trainiert allerdings nur
auf das Nachzeichnen von Ziffern. Aus diesem Grund trainiert diese Variation nur
auf die Erkennbarkeit von Ziffern, und lässt die anderen Arten von Strichbildern
aussen vor.
 
Die Reward-Function (siehe \doubleref{sub:t_rl_func}) dieser Variation beinhaltet
eine zweite KI, die handgeschriebene Ziffern erkennt (siehe
\autoref{tab:models}). Diese zweite KI beurteilt in jedem Step, welche
Ziffern sie in der Vorlage und in der aktuellen
Zeichnung erkennt.
 
Die einfachste Form der Reward-Function für diese Variation sähe
folgendermassen aus: Wenn eine Zeichnung das Kriterium der Erkennbarkeit
erfüllt, löst die Reward-Function einen Reward von $0.05$ aus. In diesem Zustand
funktioniert die Reward-Function allerdings nicht. Der Agent kann den
akkumulierten Reward nicht maximieren. Zwei Ansätze gehen auf dieses Problem
ein. Beide Ansätze sind Teil dieser Variation.
 
Der erste Ansatz schlägt vor, die zweite KI erst ab einer gewissen prozentualen
Übereinstimmung (siehe \doubleref{sub:m_eval_proc}) einzusetzen. In diesem Fall
löst die korrekte Erkennung erst ab einer prozentualen Übereinstimmung von
$20\%$ einen positiven Reward aus. Diese zusätzliche Bedingung ist notwendig,
weil die Beurteilungen der zweiten KI teilweise für einen menschlichen
Betrachter fragwürdig sind. Zum Beispiel schätzt die zweite KI eine leere
Zeichenfläche mit einer hohen Sicherheit als eine Eins ein (siehe
\autoref{fig:wrong-mnist-rec}). Das ist ein Problem, weil dadurch der Agent
einen positiven Reward für eine leere Zeichenfläche erhält. Das stört das
weitere Lernverhalten, weil es die Wahrscheinlichkeit erhöht, dass die KI nicht
mehr zeichnet.
 
% Bild falsche mnist rec
\begin{figure}[!ht]
 \centering
 \includegraphics[width=\textwidth-2cm]{images/methode/wrong-mnist-rec.png}
 \caption{Beispiele einer richtigen und einer falschen Erkennung von handgeschriebenen Zahlen durch eine KI (eigene Abbildung). Die Werte sind durch einen Test der KI berechnet.}\label{fig:wrong-mnist-rec}
\end{figure}
 
Der zweite Ansatz implementiert neben der Reward-Function der Erkennbarkeit
erneut die Basis Reward-Function (siehe \doubleref{sub:m_var_base}). Die Relevanz
der beiden Reward-Functions ändert sich allerdings über das Training hinweg. Die
Rewards werden in jedem Step mit einem bestimmten Faktor multipliziert. Zu
Beginn des Trainings ist der Faktor für den Reward der Basis Reward-Function
$f_b = 1$ und der Faktor für den Reward basierend auf der Erkennbarkeit $f_e =
0$. Vom Start ausgehend sinkt $f_b$ linear und $f_e$ steigt linear. Ab einem
gewissen Punkt bleiben beide Faktoren stehen (siehe \autoref{fig:decrementor}).
Blieben die Faktoren ab diesem Punkt nicht konstant, würde die Variation,
gestützt auf Beobachtungen, an der Stabilität ihrer Leistung verlieren.

\newpage

% Bild Decrementor
\begin{figure}[!ht]
 \centering
 \includegraphics[width=\textwidth-2cm]{images/methode/decrementor.png}
 \caption{Veränderung der Faktoren der Basis Reward-Function und der Reward-Function der Erkennbarkeit über das Training hinweg. (eigene Abbildung)}\label{fig:decrementor}
\end{figure}
 
Das Zusammenspiel der beiden Reward-Functions hat den Vorteil, dass die
künstliche Intelligenz zu Beginn des Trainings durch die Basis Reward-Function
für kleine Erfolge positive Rewards erzielt. Die Reward-Function der
Erkennbarkeit ermöglicht das nicht, da sie erst für eine korrekte Erkennung
einen Reward auslöst. Eine korrekte Erkennung ist allerdings für eine
untrainierte KI schwer zu erreichen. Gewissermassen wird die KI durch die Basis
Reward-Function vortrainiert, um schlussendlich mit der Reward-Function der
Erkennbarkeit effizient trainieren zu können.
 
 
\subsection{Physikalische Umgebung}\label{sub:m_var_phy}
Diese Variation spezialisiert sich auf kein Kriterium. Stattdessen verändert
sich die Umgebung, in der sich der Agent bewegt (siehe
\doubleref{sub:t_rl_func}). Auch der Input und der Output des neuronalen Netzes
sind angepasst. Durch diese Veränderungen löst sich die Variation vom
Grundprogramm. Sie bleibt allerdings mit den anderen Variationen kompatibel, da
diese ausschliesslich die Reward-Function anpassen.
 
Die Variation ergänzt die Umgebung durch physikalische Simulationen. Diese
physikalische Umgebung definiert die physischen Rahmenbedingungen des Zeichnens
neu, mit dem Ziel, diese näher an die Realität zu bringen.
 
Der Agent hat neu eine Geschwindigkeit, die durch einen Vektor $\vec{v}$
dargestellt ist. Die Geschwindigkeit beschreibt, um wie viele Pixel und in
welche Richtung sich der Agent pro Step bewegt. Die folgende Formel beschreibt,
wie sich die Position des Agenten vom Step $t$ bis zum nächsten Step $t+1$
verändert:
\[ \vec{p}(t+1) = \vec{p}(t) + \vec{v}(t) \]
$\vec{p}(t)$ beschreibt die Position des Agents als einen Ortsvektor auf der
Zeichenfläche zum Step $t$ und $\vec{v}(t)$ beschreibt die Geschwindigkeit des
Agenten zum Step $t$. Die Position rundet in jedem Step auf ganze Zahlen. Das
kommt daher, dass die Geschwindigkeit auch Dezimalzahlen annehmen kann, aber die
Position nur durch ganze Zahlen dargestellt ist.
 
Zur Geschwindigkeit des Agents wird in jedem Step ein Beschleunigungsvektor
addiert. Jede Action, die der Agent wählen kann, entspricht einem anderen
Beschleunigungsvektor. Der Action-Space (siehe \doubleref{sub:t_rl_func})
besteht neu aus $42$ Actions. $21$ der $42$ Actions beschreiben
Beschleunigungsvektoren im zeichnenden Zustand. Die anderen $21$ Actions
beschreiben dieselben Vektoren im nicht zeichnenden Zustand. Die 21
verschiedenen Beschleunigungsvektoren im Actions-Space sind in der folgenden
Formation angeordnet: (siehe \autoref{fig:physics-actionspace}).
 
%bild physik actionspace
\begin{figure}[!ht]
 \centering
 \includegraphics[width=\textwidth-6cm]{images/methode/physics-actionspace.png}
 \caption{Action-Space in der physikalischen Umgebung. (eigene Abbildung)}\label{fig:physics-actionspace}
\end{figure}
 
 
Mit dem gewählten Beschleunigungsvektor $\vec{a}(t)$ berechnet sich die
Geschwindigkeit im nächsten Step $t+1$ aus dem aktuellen Step $t$ durch folgende
Formel:
\[ \vec{v}(t+1) = \vec{v}(t) + \vec{a}(t) \] Der Betrag der Geschwindigkeit
$\vec{v}(t+1)$ des Agents wird in jedem Step, unabhängig von der gewählten
Action, um $0.3$ Pixel pro Step verringert. Das simuliert eine Reibungskraft,
die auf den Agent einwirkt.
 
 
Die Veränderungen in der Umgebung erfordern Anpassungen im neuronalen Netz
(siehe \doubleref{sub:t_ml_nn}). Ohne diese Anpassungen kann die KI den
akkumulierten Reward nicht maximieren. Das Problem ist, dass die aktuelle
Geschwindigkeit des Agents kein Teil der Observation ist (siehe
\doubleref{sub:t_rl_func}). Der Agent berücksichtigt deswegen seine
Geschwindigkeit nicht in seinen Entscheidungen. Die Lösung dieses Problems
bietet eine Verschiebung des Local image Patches (siehe \doubleref{sub:t_ver_dood}).
Im Grundprogramm entspricht der Mittelpunkt des Local image Patches genau der
Position des Agents. Neu befindet sich der Mittelpunkt dort, wo sich der Agent
laut seiner aktuellen Geschwindigkeit im nächsten Step befinden wird. Durch
diese Verschiebung des Local image Patches erhält der Agent Informationen über
seine Geschwindigkeit, ohne dessen numerischen Wert zu kennen. Wie im
Grundprogramm gibt der Local image Patch den gesamten Bereich an, in dem sich
der Agent im nächsten Step befinden kann. Die tatsächliche neue Position des
Agents wird durch die Action seiner Wahl bestimmt. Die Grösse des Local image
Patches schrumpft von $7\times7$ Pixeln auf $5\times5$ Pixel, da alle möglichen
Positionen des Agents nach einem Step auf einem $5\times5$ Feld Platz haben
(siehe \autoref{fig:patch-move}).
 
%bild local patch verschiebung
\begin{figure}[!ht]
 \centering
 \includegraphics[width=\textwidth]{images/methode/patch-move.png}
 \caption{Angabe der Geschwindigkeit durch eine Verschiebung des Local image Patches. (eigene Abbildung)}\label{fig:patch-move}
\end{figure}
 
 
Ein weiteres Problem ist, dass der Agent sich durch seine Geschwindigkeit aus
den vorgegebenen Grenzen der Zeichenfläche begeben kann. Im Grundprogramm 
(siehe \doubleref{sub:m_grund_dood}) kann der Agent Actions, die ihn in eine
unzulässige Position bewegen würden, nicht auswählen. Wenn allerdings in der
physikalischen Umgebung die Geschwindigkeit des Agents zu hoch ist, kann dieser
keine Actions mehr wählen, die ihn innerhalb der Grenzen der Zeichenflächen
halten würden. Als Lösung wird diesen Fällen die Geschwindigkeit des Agents auf den
Nullvektor zurückgesetzt und die Reward-Function löst einen negativen Reward von
$-0.05$ aus. Der negative Reward soll die Häufigkeit dieser Vorfälle vermindern.
 
 
\section{Auswertung}\label{chap:m_auswert}
Die Auswertung der Daten über die Leistung der künstlichen Intelligenz liefert
das Resultat der Methode. Die Auswertung berechnet den Zahlenwert der
definierten Kriterien (siehe \doubleref{chap:m_eval}) für verschiedene Variationen
der KI.
 
Die Variationen werden auf ihre Leistung für drei verschiedene Datensets
geprüft. Die drei Datensets beinhalten verschiedene Arten von
Strichbildern (siehe \autoref{tab:models}). Im Falle des QuickDraw Datensets
wird die KI allerdings nur auf das Nachzeichnen von zehn Motiven überprüft. Die
zehn Motive sind: `Amboss', `Apfel', `Besen', `Eimer', `Bulldozer', `Uhr',
`Wolke', `Computer', `Auge' und `Blume' (siehe
\autoref{fig:quickdraw-examples}). Die Bilder in den drei Datensets sind gleich
verarbeitet wie die Trainingsdaten (siehe \doubleref{sub:m_grund_data}).
 
%Bild images dataset
\begin{figure}[!ht]
 \centering
 \includegraphics[width=\textwidth]{images/methode/quickdraw-examples.png}
 \caption{Beispiele der verwendeten Motive aus dem QuickDraw Datenset. (eigene Abbildung)}\label{fig:quickdraw-examples}
\end{figure}
 
 
Die Variationen (siehe \doubleref{chap:m_var}) der KI umfassen zwei Umgebungen und
drei Reward-Functions (siehe \doubleref{sub:t_rl_func}). Für jede Variation ist zur
Vereinfachung eine Abkürzung definiert.
\newpage
\begin{itemize}
 \item \doubleref{chap:m_grund} Umgebung: Grund
 \item \doubleref{sub:m_var_phy}: Physik
 \item \doubleref{sub:m_var_base}: Basis
 \item \doubleref{sub:m_var_speed}: Speed
 \item \doubleref{sub:m_var_rec} (von MNIST Ziffern): MNIST
\end{itemize}
 
Die folgenden Kombinationen an Variationen der künstlichen Intelligenz werden
ausgewertet. Diese Kombinationen stellen einzelne Versionen der KI dar
\begin{itemize}
 \item Grund-Basis
 \item Grund-MNIST
 \item Grund-Speed
 \item Grund-MNIST-Speed
 \item Physik-Basis
 \item Physik-MNIST
 \item Physik-Speed
 \item Physik-MNIST-Speed
\end{itemize}
 
\subsection{Testumgebung}\label{sub:m_auswert_test} Die Leistungen der
verschiedenen Variationen der künstlichen Intelligenz werden in einer
Testumgebung (siehe \doubleref{sub:t_ml_func}) ausgewertet. Zwischen der
Trainingsumgebung und der Testumgebung sind drei relevante Unterschiede. Erstens
trainiert die KI in der Testumgebung nicht. Die Testumgebung übernimmt eine
trainierte Version der KI und verändert diese während dem Test nicht. Zweitens
wählt der Agent in keinem Fall mehr eine zufällige Action. Stattdessen wählt er
immer die Action mit dem höchsten Q-Value (gleichbedeutend mit $\epsilon = 0$)
(siehe \doubleref{sub:t_rl_func}). Der Dritte Unterschied liegt in den
Strichbildern, die für die künstliche Intelligenz als Vorlage dienen. Im Test
zeichnet das Computerprogramm $2000$ Bilder aus einem der drei zur Verfügung
stehenden Datensets (siehe \autoref{tab:models}).
 
Am Ende jeder Episode (das heisst jeder Zeichnung), wird der Zahlenwert für die
verschiedenen Kriterien (siehe \doubleref{chap:m_eval}) ihrer Definition
entsprechend ausgewertet und gespeichert. Der Durchschnitt aller gespeicherten
Werte eines Kriteriums entspricht der Leistung der getesteten Variation in
diesem Kriterium. Das Kriterium der Erkennbarkeit verwendet zur Auswertung
dasjenige vortrainierte Modell, das auf denselben Daten trainiert ist, die im
Test verwendet werden (siehe \doubleref{sub:m_eval_rec}). Da das Kriterium der
Erkennbarkeit entweder den Wert $0$ oder $1$ hat, ergibt der Durchschnitt aus
allen Werten dieses Kriteriums eine prozentuale Angabe (in Dezimalform) darüber,
in wie vielen Fällen das richtige Motiv erkannt wird.

\chapter{Resultate}\label{chap:r}
Die Resultate bestehen aus drei Tabellen. Jede Tabelle beschreibt die Leistung
der acht Versionen (siehe \doubleref{chap:m_var}), bezogen auf die drei definierten
Kriterien (siehe \doubleref{chap:m_eval}). Die Daten in den Tabellen
stammen aus den Tests der KI (siehe \doubleref{chap:m_auswert})
Der Unterschied in den Tabellen liegt im Datenset, mit denen die Versionen der
KI jeweils getestet sind. 
Eine Sammlung von gezeichneten Strichbildern ergänzt die Resultate. Die
Strichbilder sind dabei jeweils in Paaren angeordnet. Das linke Bild im Paar
zeigt die Vorlage aus dem Datenset und das rechte Bild zeigt die nachgezeichnete
Variante von der KI. Die Zeichnungen, die in der Sammlung
vertreten sind, sind zufällig ausgewählt aus dem Test der jeweiligen Version der
KI. Die Bilder haben einen Farbverlauf, der den zeitliche
Verlauf des Zeichnens dargestellt. Die Helligkeit eines Striches ist
proportional zu dem Step, in dem dieser gezeichnet wurde. Das bedeutet, dass
dunklere Striche früher gezeichnet werden als hellere Striche. Bewegungen des
Agents, in denen dieser nicht zeichnet, sind in den Bildern nicht erkennbar.

\newpage
\section{Tabellen}\label{chap:r_tab}
\begin{table}[!ht]
    \centering
    \caption{Testen auf MNIST Datenset | 2000 Tests}\label{tab:MNIST}
    \begin{tabular}{|l|l|l|l|}
        \hline
            ~ & Übereinstimmung \% & Erkennbarkeit \% & Geschwindigkeit \\ \hline
            Grund-Basis & 86.5 & 86.6 & 24.5 \\ \hline
            Grund-MNIST & 66.8 & 64.3 & 51.2 \\ \hline
            Grund-Speed & 86.2 & 86.6 & 25.5 \\ \hline
            Grund-MNIST-Speed & 61.4 & 55.1 & 56.8 \\ \hline
            Physik-Basis & 56.4 & 46.4 & 62.5 \\ \hline
            Physik-MNIST & 38.4 & 35.7 & 63.9 \\ \hline
            Physik-Speed & 63.0 & 58.2 & 61.2 \\ \hline
            Physik-MNIST-Speed & 29.2 & 27.3 & 63.7 \\ \hline
        \end{tabular}
\end{table}

\begin{table}[!ht]
    \centering
    \caption{Testen auf EMNIST Datenset | 2000 Tests}\label{tab:EMNIST}
    \begin{tabular}{|l|l|l|l|}
    \hline
        ~ & Übereinstimmung \% & Erkennbarkeit \% & Geschwindigkeit \\ \hline
        Grund-Basis & 86.8 & 74.5 & 38.2 \\ \hline
        Grund-MNIST & 65.2 & 45.0 & 57.4 \\ \hline
        Grund-Speed & 87.8 & 77.0 & 37.7 \\ \hline
        Grund-MNIST-Speed & 62.2 & 40.0 & 60.9 \\ \hline
        Physik-Basis & 57.6 & 32.4 & 63.5 \\ \hline
        Physik-MNIST & 43.3 & 23.6 & 63.9 \\ \hline
        Physik-Speed & 56.3 & 35.0 & 63.6 \\ \hline
        Physik-MNIST-Speed & 30.2 & 13.9 & 64.0 \\ \hline
    \end{tabular}
\end{table}

\begin{table}[!ht]
    \centering
    \caption{Testen auf QuickDraw-Datenset | 2000 Tests}\label{tab:Quickdraw}
    \begin{tabular}{|l|l|l|l|}
    \hline
        ~ & Übereinstimmung \% & Erkennbarkeit \% & Geschwindigkeit \\ \hline
        Grund-Basis & 79.1 & 80.5 & 39.1 \\ \hline
        Grund-MNIST & 57.3 & 62.5 & 59.9 \\ \hline
        Grund-Speed & 79.5 & 82.2 & 40.0 \\ \hline
        Grund-MNIST-Speed & 54.9 & 58.9 & 62.5 \\ \hline
        Physik-Basis & 48.1 & 55.7 & 63.8 \\ \hline
        Physik-MNIST & 30.5 & 38.9 & 64.0 \\ \hline
        Physik-Speed & 50.0 & 58.3 & 63.6 \\ \hline
        Physik-MNIST-Speed & 22.4 & 31.1 & 64.0 \\ \hline
    \end{tabular}
\end{table}

\newpage

\section{Bildersammlung}\label{chap:r_bild}
\begin{figure}[!ht]
    \centering
    \includegraphics[width=\textwidth]{images/resultate/base-base.png}
    \caption{Grund-Basis}\label{fig:Grund-Basis}
\end{figure}

\newpage
\begin{figure}[!ht]
    \centering
    \includegraphics[width=\textwidth]{images/resultate/base-mnist.png}
    \caption{Grund-MNIST}\label{fig:Grund-MNIST}
\end{figure}

% \begin{figure}[!ht]
%     \centering
%     \includegraphics[width=\textwidth]{images/resultate/base-speed.png}
%     \caption{Grund-Speed}
%     \label{fig:Grund-Speed}
% \end{figure}

\begin{figure}[!ht]
    \centering
    \includegraphics[width=\textwidth]{images/resultate/physics-base.png}
    \caption{Physik-Basis}\label{fig:Physik-Basis}
\end{figure}

\begin{figure}[!ht]
    \centering
    \includegraphics[width=\textwidth]{images/resultate/physics-speed.png}
    \caption{Physik-Speed}\label{fig:Physik-Speed}
\end{figure}

\chapter{Diskussion}\label{chap:d}
Die Diskussion analyisiert die Resultate der Methode (siehe \doubleref{chap:m}),
um daraus eine Antwort auf die Fragestellung zu bilden. Zu diesem Zweck werden
einige allgemeine Feststellungen getroffen und die Unterfragen beantwortet
(siehe \doubleref{chap:d_frage}). Im zweiten Teil der Diskussion folgt ein Fazit,
ein Ausblick (siehe \doubleref{chap:d_faz-aus}), und eine Selbstreflexion (siehe
\doubleref{chap:d_reflex}). Dabei verschiebt sich der Fokus von der Fragestellung
weg und auf eine allgemeinere Betrachtung der Arbeit.


\section{Fragestellung und Unterfragen}\label{chap:d_frage}
Die Fragestellung und die Unterfragen decken nicht alle Erkenntnisse aus den
Resultaten ab. Einige allgemeine Feststellungen geben Einblick, wie die
Resultate (siehe \doubleref{chap:r}) zu verstehen sind. 

Die Grund-Basis Version und die Grund-Speed Version (siehe
\doubleref{chap:m_auswert}) erreichen in allen Kriterien für alle Datensets die
beste Leistung. Die Resultate zwischen den Versionen sind dabei fast
ununterscheidbar. Vor allem im Kriterium der Geschwindigkeit (siehe
\doubleref{sub:m_eval_speed}) zeigt die Speed Variation keine Verbesserung. Unter
den Versionen, die auf der physikalischen Umgebung basieren, erreichen Ebenfalls
die Physik-Basis und die Phyisk-Speed Versionen die beste Leistung. Die
Grund-MNIST Version und die Physik-MNIST Version sind in allen Kriterien
schlechter als die Basis und Speed Versionen. Auch im Kriterium der
Erkennbarkeit (siehe \doubleref{sub:m_eval_rec}) bringt die Variation keinen
Vorteil. Die Physik-MNIST-Speed Version erbringt die schlechteste Leistung. Eine
Erklärung dafür ist, dass diese Version eine Kombination von allen Variationen
(siehe \doubleref{chap:m_var}) ist und somit von der besten Version, der
Grund-Basis Version am stärksten abweicht.

Die Bildersammlung (siehe \doubleref{chap:r_bild}) zeigt, dass die KI in vielen
Fällen präzise der Linie folgt und selten willkürliche Sprünge begeht. Das ist
interessant, weil der KI nie explizit mitgeteilt wird, wie genau sie zeichnen
sollte.




\subsection{Beantwortung der Unterfragen}\label{sub:d_frage_unter}
Insgesamt sechs Unterfragen werden beantwortet (siehe \doubleref{chap:einleit}).
Diese Unterfragen weiten die Fragestellung aus und tragen zu der
schlussendlichen Antwort auf die Fragestellung bei. Die Antworten beruhen auf
den Resultaten, aber auch auf Erkenntnissen aus der Methode selbst (siehe
\doubleref{chap:m}).

\subsubsection*{Wie kann die Architektur einer KI aussehen, die das Nachzeichnen erlernt?}\label{subsub:d_frage_unter_1}
Unter der Annahme, dass die KI dieser Arbeit das Nachzeichnen erlernt, (siehe
\doubleref{sub:d_frage_frag}), kann die Architektur genau so aussehen, wie sie in
dieser Arbeit beschrieben ist (siehe \doubleref{sub:m_grund_dood}).

\subsubsection*{Wie lässt sich die Leistung der KI in ihrer Aufgabe beurteilen?}\label{subsub:d_frage_unter_2}
Die Leistung der KI lässt sich durch die definierten Kriterien (siehe
\doubleref{chap:m_eval}) beurteilen. Das Kriterium Die Übereinstimmung ist ein
objektiver und Absoluter Wert, und somit das Aussagekräftigste Kriterium.
Ausserdem ist der maximale Wert des Kriteriums, unabhängig vom gezeichneten
Bild, gleich $1$. Dadurch ist das Kriterium geeignet für Vergleiche zwischen
Versionen der KI.

die Kriterien der Erkennbarkeit und der Geschwindigkeit sind an subjektive
Annahmen gebunden. Zum Beispiel wird für das Kriterium der Geschwindigkeit ein
subjektiver Punkt der Fertigstellung definiert (siehe
\doubleref{sub:m_eval_speed}). Dadurch sinkt ihre Aussagekraft. Allerdings
verändern sich die Annahmen nicht und die Kriterien sind in jedem Fall durch
einen Zahlenwert repräsentiert. Somit eignen sich auch diese Kriterien für
Vergleiche zwischen Versionen der KI. Aus der Annahme heraus, dass für Menschen
beim Nachzeichnen Erkennbarkeit wichtiger als absolute Genauigkeit ist, ergibt
sich das Kriterium der Erkennbarkeit als besonders wichtiges. Aus diesem Grund
ist das Kriterium in der Fragestellung (siehe \doubleref{sub:d_frage_frag})
vermerkt.


\subsubsection*{Wie lässt sich die Leistung der KI in ihrer Aufgabe verbessern?}\label{subsub:d_frage_unter_3}
Bezogen auf die definierten Kriterien erreicht die Grundversion Werte, die durch
die implementierten Variationen nicht oder nur marginal verbessert werden. Die
Variationen der KI sind somit insgesamt ein gescheiterter Versuch der
Verbesserung der Leistung. Die Grundversion erfuhr allerdings in dessen
Entwicklung signifikante Verbesserungen. Die grössten Verbesserungen stammen aus
der Optimierung der Hyperparamter durch den Baysian Optimization Algorithmus
(siehe \doubleref{sub:m_grund_data}). Zum Beispiel hat die Grösse des Replay
Buffers einen erheblichen Effekt auf die Leistung.

\subsubsection*{Welche Einflüsse haben physische Einschränkungen auf die Leistung der KI?}\label{subsub:d_frage_unter_4} 
Die phyischen Einschränkungen, die auf Physiksimulationen basieren (siehe
\doubleref{sub:m_var_phy}), verschlechtern die Leistung der KI in den
definierten Kriterien. Alle Versionen, die auf der Grundumgebung basieren,
erzielen höhere Werte als die gleichen Versionen basierend auf der
phyiskalischen Umgebung. Die physikalische Umgebung hat zum Ziel, die
Bewegungen der KI realistischer zu gestalten. In diesem Bereich kann der
Einfluss nicht objektiv bestimmt werden. Aus Beobachtungen der Bilder, welche in
der physikalischen Umgebung gezeichnet sind (siehe \doubleref{chap:r_bild}),
gehen ebenfalls keine Erkenntnisse in diesem Bereich hervor. Die Bilder
unterscheiden sich kaum von denjenigen aus der Grundumgebung.

\subsubsection*{Wie ändert sich die Leistung der KI für Strichbilder, die im Training nicht enthalten sind?}\label{subsub:d_frage_unter_5} 
In allen acht Versionen bleibt die Leistung der KI zwischen den drei Datensets
(siehe \autoref{tab:modelle}) vergleichbar. Die Tabellen (siehe
\doubleref{chap:r_tab}) zeigen die Leistung der Grund-Basis Version und der
Physik-Basis Version in den drei definierten Kriterien, getestet auf die drei
Datensets. Der Wert der Übereinstimmung zwischen dem MNIST Datenset und dem
EMNIST Datenset ist beinahe identisch. Für beide Versionen ist der Wert der
Übereinstimmung für das QuickDraw Datenset niedriger. Insgesamt ist die KI in
diesem Kriterium jedoch kaum Beeinflusst durch die Wahl des Datensets. Die
Analyse der anderen zwei Kriterien führt zu einer ähnlichen Schlussfolgerung.
Interessant ist, dass vor allem die Grund-Basis Version eine viel höhere
Geschwindigkeit im Zeichnen von MNIST Zahlen hat, als im Zeichnen von EMNIST
Buchstaben. Obwohl die Formen zu grossem Teil ähnlich sind, scheint die KI durch
das spezifische Training auf MNIST Ziffern eine höhere Geschwindigkeit zu
entwickeln.

\subsubsection*{Wie und in wiefern lässt sich das Verhalten der KI mit menschlichem Zeichnen vergleichen?}\label{subsub:d_frage_unter_6}
Die Antwort auf diese Frage leitet sich nicht aus den objektiven Resultaten ab,
sondern basiert auf subjektiven Beobachtungen. Die Bewegungen in der
Physik-Version der künstlichen Intelligenz basieren grundsätzlich auf den selben
Gesetzen wie die Bewegungen in der echten Welt. Allerdings sind die Bewegungen
stark vereinfacht im Vergleich zu menschlichen Bewegungen Ausserdem ist für die
künstliche Intelligenz der Druck des Stiftes nicht veränderbar. Zumindest
Konzeptuell nähert die künstliche Intelligenz menschliches Zeichnen,
bezogen auf die physischen Einschränkungen, an. Einige menschliche Gewohnheiten
sind bei der künstlichen Intelligenz allerdings nicht beobachtbar. Zum Beispiel
beginnt die künstliche Intelligenz beim Zeichnen von Ziffern an zufälligen
Orten, während Menschen in der Regel für jede Ziffer an der selben Stelle
ansetzen einer Ziffer immer an der selben Stelle


\subsection{Beantwortung der Fragestellung}\label{sub:d_frage_frag}
Die Fragestellung lautet: In wiefern kann eine künstliche Intelligenz lernen,
Strichbilder auf eine physische Weise nachzuzeichnen, sodass diese durch ein
automatisches System erkannt werden? (siehe \doubleref{chap:einleit}) Diese Frage
hat mehrere Aspekte, die teilweise bereits durch die Unterfragen (siehe
\doubleref{sub:d_frage_unter}) erfasst werden. Für die schlussendliche Antwort
folgt eine genauere Ausführung der Aspekte.

Die KI zeichnet durch Physiksimulationen und durch allgemeine Einschränkungen
der Bewegungsfreiheit auf eine annähernd physische Weise. Das Zeichnen ist nur
annähernd physisch, da alle Bewegungen simuliert und in keiner phyischen
Umgebung umgesetzt sind. Ausserdem sind die Simulationen nicht vollkommen
realitätsgetreu (siehe \ref{sub:d_frage_unter} \doubleref{subsub:d_frage_unter_4})

Die künstliche intelligenz erlernt das Nachzeichnen bezogen auf die Kriterien,
nach denen es definiert ist, erfolgreich. Dafür sprechen die Werte der besten
Versionen für das Nachzeichnen von Ziffern, die teilweise an den Höchstwert
grenzen (siehe \doubleref{chap:d_frage}). Die hohen Werte im Kriterium der
Erkennbarkeit bestätigen ausserdem, dass die Zeichnungen der KI in den meisten
Fällen von einem automatischen System erkannt werden.

Laut der Fragestellung soll die KI das Nachzeichnen von Strichbildern erlernen.
Damit ist implizit das Nachzeichnen von allen möglichen Arten von Strichbildern
gemeint. Die Leistung der KI kann nicht auf alle möglichen Strichbilder
überprüft werden, aber der Test mit drei verschiedenen Datensets ergibt
vielversprechende Resultate (siehe \doubleref{chap:r_tab}). Die KI erlernt
erfolgreich das Nachzeichnen von Ziffern, Kleinbuchstaben und zehn zufälligen
Motiven aus dem QuickDraw Datenset. Durch die Vielfalt im QuickDraw Datenset
kann die Annahme getroffen werden, dass die KI zumindest einen grossen Teil an
Strichbildern nachzeichnen kann. 

Die zusammenfassende Antwort auf die Frage lautet somit: Eine künstliche
Intelligenz kann das Nachzeichnen von Strichbildern auf annähernd physiche Weise
in dem Sinne lernen, dass die fertige Zeichnung von einem automatischen System
grösstenteils erkannt wird, die Übereinstimmung zwischen der Vorlage und der
Zeichnung gross ist und die Zeichnung nicht viel Zeit in Anspruch nimmt.

Diese Antwort bezieht sich auf die genaue Frage, wie sie in der Einleitung steht.
Der nächste Abschnitt beurteilt die Frage durch die Erkenntnisse aus dieser
Arbeit und geht auf mögliche Erweiterungen ein.


\section{Fazit und Ausblick}\label{chap:d_faz-aus}
Die Resultate erlauben eine positive Antwort auf die Fragestellung. Diese
Antwort setzt allerdings einige Annahmen vorraus, die weiter diskutiert werden
können. Die grösste Annahme bezieht sich auf die Definition des Nachzeichnens.
Diese Arbeit definiert Nachzeichnen durch drei Kriterien und durch physische
Rahmenbedingungen. Die Kriterien sind für eine künstliche Intelligenz sinnvoll
gewählt (siehe \doubleref{subsub:d_frage_unter_2}), allerdings wären auch ander
Kriterien möglich. Nachzeichnen ist eine menschliche Tätigkeit. Dieser
menschliche Aspekt ist in den definierten Kriterien nicht enthalten.

Die physischen Rahmenbedingungen unterscheiden sich von denjenigen, die ein
Mensch erfährt. Das kommt daher, dass die phyischen Rahmenbedingungen für die KI
lediglich simuliert sind. Das verunmöglicht eine umfassende Antwort auf die
Frage, ob die künstliche Intelligenz auf eine physische Weise zeichnet. Dieses
Problem könnte mit einem Roboter gelöst werden, der die künstliche Intelligenz
in eine reale, phyische Umgebung überführt. Der Roboter könnte somit
verschiedenste Strichbilder auf einem echten Stück Papier, und somit
zwangsläufig auf physische Weise nachzeichnen. Aktuell sind die Bewegungen der
künstlichen Intelligenz in gewissen Belangen eingeschränkt. So ist
beispielsweise die Druckstärke nicht variierbar. Ausserdem zeichnet die
künstliche Intelligenz vorwiegend kleine Strichbilder. Experimente mit grösseren
Konstrukten, wie ganze Wörter, wären eine mögliche Erweiterung. 

Alles in allem sind eine Vielzahl an denkbaren Fragen und Ideen möglich, die auf
ReSketch, der künstlichen Intelligenz hinter dieser Arbeit, basieren.


\section{Selbstreflexion}\label{chap:d_reflex} Die Selbstreflexion gibt genauere
Einblicke in die Vorangehensweise hinter dieser Arbeit. Diese Dokumentation ist
grundsätzlich eine Zusammenfassung der wichtigsten Ereignisse. Viele Aspekte,
wie auch die Arbeitsweise bleiben verschwiegen. Die Selbstreflexion geht näher
auf drei wichitige Aspekte ein, die in der zusammengefassten Dokumentation nicht
genug betont sind.

Die Dokumentation ist mit LaTeX und spezifischer der ETH Thesis Formatvorlage
\cite{noauthor_cadmo_2014} formatiert. Ein Grossteil der Abbildungen stammt von den Autoren
und ist in Adobe Illustrater oder der Python Library Matplotlib erstellt

\subsection{Optimierung der KI}\label{sub:d_reflex_opti} Insgesamt sind acht
Versionen der KI präsentiert. Im Verlaufe des Projektes gab es viele weitere
Versuche, die Leistung der künstlichen Intelligenz zu verbessern. Diese Versuche
führten allerdings häufig dazu, dass die KI den akkumulierten Reward (siehe
\doubleref{sub:t_rl_func}) nicht mehr maximieren konnte. In der Dokumentation sind
deswegen nur diejenigen Versuche vermerkt, die tatsächlich funktionieren. Das
Problem hinter den versuchten Variation liegt darin, dass die Ursache hinter
ihrem Scheitern oder ihrem Erfolg häufig nicht erkennbar ist. Das macht die
Optimierung der künstlichen Intelligenz allgemein schwierig. 

Die Strategie hinter der Optimierung besteht in den meisten Fällen aus
wiederholtem Ausprobieren mit Anpassungen zwischen jedem Versuch. Hilfsmittel,
wie der Baysian Optimization Algorithmus (siehe \doubleref{sub:t_ml_hyper}),
vereinfachen diese Aufgabe massgeblich. Tatsächlich ermöglichte der Baysian
Optimization Algorithmus eine Verbesserung der KI von ungefähr $40-50\%$ mehr im
Kriterium der Übereinstimmung (siehe \doubleref{sub:m_eval_proc}). Diese Strategie
der Optimierung ist für einen Computer sehr ressourcenintensiv. In den längsten
Optimierungsarbeiten liefen die beiden Computer, auf denen die Arbeit verrichtet
wurde, zusammen länger als $48$ Stunden.

\subsection{Analyse der KI}\label{sub:d_reflex_analys} Eine
Analyse der künstlichen Intelligenz ist notwendig, um die Fragestellung und die
Unterfragen zu beantworten. Aber auch während der Entwicklung der KI ist eine
stetige Analyse nötig, um dise zu verstehen und zu verbessern.

Die Analyse besteht hauptsächlich darin, die Leistung der künstlichen
Intelligenz zu beurteilen. Das geschieht mittels den Kriterien, die für diesen
Zweck definiert sind (siehe \doubleref{chap:m_eval}). Die Kriterien sind dabei so
definiert, dass sie für jede mögliche Variation identisch bleiben. Der
durchnittliche akkumulierte Reward ist beispielsweise absichtlich kein
Kriterium. Der akkumulierte Reward ist abhängig von der Reward-Function (siehe
\doubleref{sub:t_rl_func}) und unterscheidet sich somit zwischen Variationen.
Dieser ist somit nicht für einen Vergleich zwischen Variationen geeignet.

% Abschnitt wäre Streichbar
Einzelne Variationen vergleichbar zu halten, ist ein allgemeines Ziel der
Analyse. Deswegen basieren alle Variationen auf der gleichen Architektur der KI.
Die Funktionalität des Grundprogrammes (siehe \doubleref{chap:m_grund}) ist
gründlich getestet. Zum Beispiel sind die Testdaten darauf überprüft, dass sie
keine Datenpunkte aus den Trainingsdaten beinhalten. Diese Tests eliminieren
mögliche Fehlerquellen falls die künstliche Intelligenz unerwartetes Verhalten
aufzeigt.

Eine weitere Form von Analyse stammt aus der Sammlung von Daten über das
Lernverhalten der KI. So wird aus jedem Training ein Graph erstellt, der die
durchschnittliche Leistung der KI in jeder Episode erfasst (siehe
\autoref{fig:learnplot}). Die Leistung ist dabei durch den akkumulierten Reward
in jeder Episode repräsentiert. Wie erwähnt können Versionen der KI nicht anhand
ihres akkumulierten Rewards verglichen werden. Der akkumulierte Reward zeigt
allerdings für einzelne Versionen am präzisesten, in wiefern diese ihren Reward
maximieren können.

%bild Lernkurve
\begin{figure}[!ht]
    \centering
    \includegraphics[width=\textwidth-2cm]{images/diskussion/learnplot.png}
    \caption{Akkumulierter Reward zu jeder Episode (Lernverhalten) der Grund-Basis Version und der Physik-Basis Version (eigene Abbildung)}
    \label{fig:learnplot}
\end{figure}


\subsection{Verwendung von Git und GitHub}\label{sub:d_reflex_git} Die
Verwendung von Git und Github (siehe \doubleref{chap:t_git}) erleichtert die
Arbeit an einem Projekt von dieser Grösse massgelich. Die Programme ermöglichen
einfache Zusammenarbeit am Programmcode und an der Dokumentation. GitHub dient
dabei zusätzlich als Hilfsmitel zur Organisation durch die integrierte Funktion
der Project Boards. Diese Funktion hätte allerdings zu grösserem Ausmass
Verwendung finden können.

Die Funktion der Branches und Commits von Git werden durch die Arbeit hindurch
konsequent verwendet. Dabei wird die Giflow Arbeitsweise, abgesehen von den
Release Branches und den Hotfix Branches, angewendet (siehe
\doubleref{sub:t_git_git}) Neben den Feature Branches werden ausserdem
Dokumentation Branches eingeführt, in denen jeweils ein Kapitel der
Dokumentation verfasst wird. Für die Zusammenführungen der wichtigsten Branches
wird das Prinzip der Pull Request (siehe \doubleref{sub:t_git_gh}) angewendet. Die
Pull Request muss für jeden Branch von beiden Autoren akzeptiert werden.

Ein weiterer Vorteil von Git und Github ist die Zugänglichkeit des Projektes.
Das gesamte Projekt ist unter folgendem Link einsehbar:
\url{https://github.com/LarsZauberer/Nachzeichner-KI}. Im Projektordner sind
vortrainierte Variationen der künstlichen Intelligenz enthalten. Das Projekt auf
GitHub erfährt möglicherweise Erweiterungen, die in dieser Arbeit nicht mehr
erfasst sind.

\chapter{Zusammenfassung}\label{zusammenfassung}
Diese Untersuchung beantwortet die Frage, inwiefern eine künstliche Intelligenz
Strichbilder auf eine physische Weise nachzeichnen kann, sodass diese durch ein
automatisches System erkannt werden. Mit Strichbildern sind in diesem
Fall Ziffern aus dem MNIST Datenset, Buchstaben aus dem EMNIST Datenset und
weitere Motive aus dem QuickDraw Datenset gemeint. 

Zur Beantwortung der Fragestellung wird der Begriff des Nachzeichnes definiert.
Zu der Defintion gehören die Rahmenbedingungen, nach denen eine Tätigkeit als
Nachzeichnen gilt, und die Kriterien, die die Leistung im Nachzeichnen
beurteilen. Zu den Rahmenbedingungen gehören unter anderem die phyischen
Einschränkungen und die ausführbaren Aktionen der KI. Um die Leistung der KI im
Nachzeichnen zu bewerten, sind drei Kriterien definiert: Die Übereinstimmung der
Pixel, die Erkennbarkeit der Zeichnung und die Geschwindigkeit des Zeichnens.
Die Erkennbarkeit der Zeichnung wird durch eine zweite künstliche Intelligenz
ermittelt.

Das Ziel ist es, eine künstliche Intelligenz zu entwickeln, welche die gesetzten
Rahmenbedingungen erfüllt und eine möglichst gute Leistung nach den definierten
Kriterien erzielt. Bei der Grundsätzlichen Architektur des KI handelt es sich um
ein Deep Q-Learning Modell, das auf der Arbeit hinter `Doodle-SDQ' \cite{zhou_learning_2018}
basiert.

Für die Rahmenbedingungen gibt es zwei Ansätze: eine Grundversion und eine
physikalische Version. In der Grundversion kann sich die KI schrittweise um eine
begrenzte Anzahl Pixel auf einer Zeichenfläche fortbewegen. Ausserdem startet
die KI auf einer zufälligen Position auf der Zeichenfläche. Die physikalische
Version ist von simulierter Physik begleitet. So kann die KI durch
Beschleunigungen ihre aktuelle Geschwindigkeit anpassen und sich so fortbewegen,
während diese durch simulierte Reibung kontinuierlich abgebremst wird. 

Für die KI existieren weitere Variationen, die dessen Leistung nach einem
bestimmten Kriterium verbessern sollen. So existiert ein spezifisches Training
auf eine verbesserte Erkennbarkeit und Geschwindigkeit der KI. Durch
Kombinationen der Variationen und der Rahemenbedingungen existieren
schlussendlich acht Versionen der künstlichen Intelligenz.

Die acht Versionen der künstlichen Intelligenz sind alle auf das Nachzeichnen
von Ziffern trainiert. Ein Experiment bestimmt, ob diese Versionen das
Nachzeichnen allgemein erlernen. Die Leistung der Versionen wird auf das
Nachzeichnen von Strichbildern aus dem Quickdraw und dem EMNIST Letters Datenset
gemessen. Wenn die Leistung für diese Strichbilder vergleichbar bleibt mit der
Leistung für die Trainingsdaten, ermöglicht das eine positive Antwort für
die Fragestellung.

Einge Versionen der künstlichen Intelligenz zeigen hierbei vielversprechende
Ergebnisse. Die Grundversion, ohne weitere Variationen, zeichnet in $91\%$ der
Fälle eine erkennbare Ziffer, in $70\%$ der Fälle einen erkennbaren Buchstaben,
und in $72\%$ der Fälle ein erkennbares Motiv aus dem QuickDraw Datenset. 




%% Your Appendix
\appendix

\backmatter

% \bibliographystyle{plain}
% \bibliography{refs}

\printbibliography[heading=bibintoc]

\begin{KeepFromToc}
\listoffigures
\listoftables
\end{KeepFromToc}

\end{document}
