\documentclass{article}
\usepackage[utf8]{inputenc}
\usepackage[utf8]{inputenc}
\usepackage{german,a4wide,theorem}
\usepackage{graphicx} % for includegraphics
\usepackage{geometry}
\usepackage[hidelinks]{hyperref}
\usepackage{cleveref}
\usepackage{float}
\usepackage[citestyle=numeric, bibstyle=numeric]{biblatex}
\geometry{a4paper,left=20mm,right=20mm, top=3cm, bottom=2.5cm}

\def\frax{\displaystyle\frac} % Change Frac Style

\addbibresource{bib.bib}
\graphicspath{{./}}

\title{\textbf{Nachzeichner KI}}
\author{\textbf{Ian Wasser, Robin Steiner}}
\date{April 2022}

\begin{document}

\maketitle

\tableofcontents

\pagebreak

\section{Thematische Beschreibung}
\label{chap:thematische-beschreibung}
In unserem Projekt wollen wir uns mit Künstlicher Intelligenz (KI) beschäftigen.
Genauer mit Reinforcement Learning (RL). Künstliche Intelligenz ist ein sehr
spannendes Thema, da man noch nicht alles darüber weiss und es auch eine Art
Black Box ist.

In unserem Projekt möchten wir uns spezifisch um folgende Fragestellung kümmern:
`Können wir eine künstliche Intelligenz entwickeln, welche menschenähnlich
Symbole in einer Simulation, der echten Welt, nachzeichnen kann?'

Dabei spielen folgende Unterfragen eine grosse Rolle.

\begin{itemize}
    \item Wie schaffen wir es, dass die KI, überhaupt etwas irgendwie nachzeichnet?
    \item Was heisst genau menschenähnlich?
    \item Wie schaffen wir es die KI, dazu zu bekommen, diese menschenähnlichen
    Faktoren zu übernehmen?
    \item Welche physikalischen Faktoren der echten Welt sind ausschlaggebend
    für das Zeichnen?
    \item Wie können wir diese physikalischen Merkmale simulieren?
    \item Wie können wir die menschenähnlichen Faktoren und die Physik des
    Zeichnens für die KI kombinieren?
\end{itemize}

\section{Wissensstand, mögliche Quellen}
\label{chap:wissensstand}

Unser Wissenstand ist recht begrenzt auf die Grundlagen von maschinellen
Lernens. Um unser Wissen in die richtige Richtung weiter zu entwickeln, haben
wir ein Vorprojekt gemacht.

\subsection{Vorprojekt}
In unserem Vorprojekt haben wir uns besonders auf ein wissenschaftliches Projekt
aus China gestützt. \url{https://arxiv.org/abs/1810.05977}. Dort wird ziemlich
genau beschrieben, wie man eine KI dazu bringt zu zeichnen. Dieses und auch
viele weitere Quellen, wie auch die Tensorflow-Dokumentation brachten uns die
Grundlagen von Reinforcement Learning bei wodurch wir beginnen konnten mit
eigenen Experimenten.

Wir konnten die Grundlagen von dem Projekt reproduzieren und sind auf mässige
Ergebnisse gekommen. Dennoch ist unser Wissen durch das Vorprojekt so weit
gestiegen, dass wir noch viele weitere Methoden und Ideen haben, welche wir
ausprobieren können um die KI noch schlauer zu machen.

\section{Methode}
\label{chap:methode}


\subsection{Dokumentation}
\label{chap:m_dokumentation}
Die Dokumentation wird immer nebenbei entwickelt, damit dort die volle Theorie
und unser Vorgehen beschrieben ist. Diese Dokumentation wird auch auf GitHub
verfügbar sein.


\section{Ressourcen}
\label{chap:ressourcen}


\section{Ergebnis}


\section{Zeitplan}
\label{chap:zeitplan}
\begin{table}[H]
    \begin{tabular}{ll}
    \textbf{Datum} & \textbf{Beschreibung}                                                                         \\
    16.09.2021     & PV V1.0 fertig                                                                                \\
    30.09.2021     & Planung und Prototypen bauen abgeschlossen + PV V2.0 fertig                                   \\
    07.10.2021     & Prototypen Zusammenführung fertig                                                             \\
    08.10.2021     & Inhaltsverzeichnis fertig                                                                     \\
    02.12.2021     & Beta (Spielbares Spiel, welches gut hergezeigt werden kann) + Erste Version der Dokumentation
    \end{tabular}
\end{table}

\section{Bewertungskriterien}
\subsection{Industrie Standard Collaboration Tools: Git, GitHub}
\label{chap:git_github}

\subsection{LaTex Dokumentation}

\subsection{Rocket League ähnliche Systeme}

\section{Unterschrift}
\label{chap:unterschrift}

Hiermit wird genehmigt, dieses Projekt im Rahmen des Projektunterrichts
durchzuführen.

\vspace*{1cm}

Unterschrift: \hrulefill Nicolas Ruh \vspace*{2cm}

Unterschrift: \hrulefill Ian Wasser \vspace*{2cm}

Unterschrift: \hrulefill Robin Steiner \vspace*{2cm}

\printbibliography[heading=bibintoc]

\end{document}