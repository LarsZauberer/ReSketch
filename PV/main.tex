\documentclass{article}
\usepackage[utf8]{inputenc}
\usepackage[utf8]{inputenc}
\usepackage{german,a4wide,theorem}
\usepackage{graphicx} % for includegraphics
\usepackage{geometry}
\usepackage[hidelinks]{hyperref}
\usepackage{cleveref}
\usepackage{float}
\usepackage[citestyle=numeric, bibstyle=numeric]{biblatex}
\geometry{a4paper,left=20mm,right=20mm, top=3cm, bottom=2.5cm}

\def\frax{\displaystyle\frac} % Change Frac Style

\addbibresource{bib.bib}
\graphicspath{{./}}

\title{\textbf{Nachzeichner KI}}
\author{\textbf{Ian Wasser, Robin Steiner}}
\date{April 2022}

\begin{document}

\maketitle

\tableofcontents

\pagebreak

\section{Thematische Beschreibung}
\label{chap:thematische-beschreibung}
In unserem Projekt versuchen wir folgende Fragestellung zu beantworten: `Kann
eine künstliche Intelligenz in einer Simulation Symbole auf menschliche Weise
nachzeichnen'

Das Projekt fällt in den Bereich der Künstlichen Intelligenz, weil menschliche
Bewegungen nicht algorithmisch aufgefasst werden können. Ausserdem beschäftigen
wir uns mit Reinforcement learning, einer Untergruppe der künstlichen
Intelligenz

Folgende Unterfragen sollen mit dem Projekt beantwortet werden.

\begin{itemize}
    \item Wie wird eine KI allgemein aufgebaut, die das Nachzeichnen erlernt
    \item Was bedeutet es, auf menschliche Weise zu zeichnen
    \item Welche physikalischen Faktoren sind für das Zeichnen relevant.
    \item Wie muss eine Simulation aufgebaut sein, um realistische Bewegungen zu ermöglichen
    \item Wie lernt die KI im Zusammenhang mit diesen physikalischen Faktoren
    \item Welche Faktoren wirken positiv, welche negativ auf die Leistung der KI
\end{itemize}

\section{Wissensstand, mögliche Quellen}
\label{chap:wissensstand}

Unser Wissenstand ist begrenzt auf die Grundlagen des maschinellen
Lernens. Um unser Wissen in die richtige Richtung weiter zu entwickeln, haben
wir ein Vorprojekt gemacht.

\subsection{Vorprojekt}
Das Vorprojekt stützt besonders auf einer wissenschaftlichen Arbeit aus China.
\url{https://arxiv.org/abs/1810.05977}. Diese Arbeit beschreibt genau die
Architektur einer KI, die das Nachzeichnen erlernt. Unser erstes Unterziel wird
dadurch grösstenteils beantwortet, wodurch wir eine gute Grundlage erarbeiten
konnten, um den Rest des Projekts anzugehen.

Wir haben die Arbeit selbst repliziert und so Erfahrung mit der Technologie und
Hilfsmitteln wie Tensorflow gesammelt.

Die Replikation erzielt zum Zeitpunkt der Projektvereinbarung ansatzweise
ähnliche Ergebnisse wie das Original. 


\section{Methode}
\label{chap:methode}

Wir werden unser Vorprojekt ausarbeiten und die Leistung davon maximieren. Wir
orientieren uns dabei weiterhin auf die bereits genannte wissenschaftliche
Arbeit. Wenn die KI stabile Ergebnisse erzielt, dient diese als Grundlage für
Experimente und Erweiterungen, die unsere restlichen Unterfragen beantworten
sollen. 


\subsection{Dokumentation}
\label{chap:m_dokumentation}
Die Dokumentation wird nebenbei entwickelt, damit dort die volle Theorie
und unser Vorgehen beschrieben ist. Diese Dokumentation wird auch auf GitHub
verfügbar sein. 



\section{Ressourcen}
\label{chap:ressourcen}

\begin{itemize}
    \item Gute Grafikkarten (GPU accelerated computing) (besitzen wir)
\end{itemize}


\section{Ergebnis}
\label{chap:ergebnis}
Das Ergebnis unseres Projektes soll ein trainiertes Tensorflow-Modell sein,
welches Symbole auf menschliche Weise nachzeichnen kann.


\section{Zeitplan}
\label{chap:zeitplan}
\begin{table}[H]
    \begin{tabular}{ll}
    \textbf{Datum} & \textbf{Beschreibung}                                                                         \\
    03.06.2022     & PV unterschrieben                                                                             \\
    24.06.2022     & Eine KI erschaffen, welche möglichst gut ein Symbol nachzeichnen kann                         \\
    10.07.2022     & Geklärt was menschenähnlich heisst                                                            \\
    07.08.2022     & Optimierung der KI auf menschenähnliches zeichnen                                             \\
    07.08.2022     & Dokumentation, bis auf noch hinzuzufügendene Teile fertig                                     \\
    21.08.2021     & Klärung der physikalischen modellierung der Simulation                                        \\
    25.09.2021     & Optimierung der KI auf physikalische Gegebenheiten                                            \\
    \end{tabular}
\end{table}

\section{Bewertungskriterien}
\subsection{Verwendung von Git und GitHub}
\label{chap:git_github}
Wir möchten Git und GitHub verwenden um unser Projekt zu organisieren und zu verwalten.

\section{Unterschrift}
\label{chap:unterschrift}

Hiermit wird genehmigt, dieses Projekt im Rahmen des Projektunterrichts
durchzuführen.

\vspace*{1cm}

Unterschrift: \hrulefill Nicolas Ruh \vspace*{2cm}

Unterschrift: \hrulefill Ian Wasser \vspace*{2cm}

Unterschrift: \hrulefill Robin Steiner \vspace*{2cm}

\printbibliography[heading=bibintoc]

\end{document}